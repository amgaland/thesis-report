\begin{abstract}
	\setcounter{secnumdepth}{0}
	Бизнесийн байгууллагуудын хувьд ажилчдын гүйцэтгэлийг үнэлэх нь байгууллагын амжилт, хөгжлийн чухал хэсэг юм. Уламжлалт арга буюу гар ажиллагаатай үнэлгээний систем нь цаг хугацаа их шаардлага гаргаж, алдаа гарах эрсдэлтэй байдаг. Иймээс орчин үеийн технологийг ашиглан автоматжуулсан, ил тод, үр дүнтэй гүйцэтгэлийн үнэлгээний систем хөгжүүлэх шаардлага бий болсон. Энэхүү тайланд Golang, Next.js, Docker зэрэг технологийг ашиглан хэрхэн ийм системийг хөгжүүлэх талаар өгүүлнэ. Тайлангийн зорилго нь системийн онолын үндэс, шаардлага, зохиомж, хэрэгжүүлэлтийг тодорхойлж, практикт хэрэглэх боломжтой шийдэл санал болгох явдал юм.

	%	\setcounter{secnumdepth}{0} reverse this command
	\setcounter{secnumdepth}{2}

\end{abstract}

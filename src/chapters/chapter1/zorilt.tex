Эдгээр зорилтууд нь ажилтны гүйцэтгэлийн үнэлгээний системийг хэрэгжүүлэх программ хангамжийн үе шаттай үйлдлүүдээс бүрдэнэ:
\begin{enumerate}
    \item Сэдэвтэй холбоотой судалгаа хийх: Ижил төстэй системүүд (жишээ нь, Interactive.mn, Asana) болон технологийн судалгааг гүйцэтгэх.
    \item Системийг хэрэглэх боломжит хэрэглэгчдийн шаардлага тогтоох: Администратор, Менежер, Ажилтан гэсэн хэрэглэгчдийн хэрэгцээ, шаардлагыг тодорхойлох.
    \item Шаардлагатай уялдуулан системийн зохиомж гаргах: Функциональ болон технологийн шаардлагад нийцүүлэн системийн архитектур, загварыг боловсруулах.
    \item Зохиомжийн дагуу системийг хөгжүүлэх: Next.js, Go (Gin, GORM), PostgreSQL, Docker зэрэг технологийг ашиглан системийг хэрэгжүүлэх.
    \item Хөгжүүлэлтийг туршиж, алдааг засаж, сайжруулах: Системийн гүйцэтгэлийг туршиж, хэрэглэгчийн саналд үндэслэн сайжруулалт хийх.
\end{enumerate}

Эдгээр зорилго, зорилтууд нь байгууллага доторх ажилтнуудын гүйцэтгэлийн хяналтыг сайжруулах, менежментийн шийдвэр гаргалтыг дэмжихэд чиглэгдэнэ.

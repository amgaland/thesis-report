%end undeslel bichne
Ажилтны гүйцэтгэлийн үнэлгээ нь байгууллагын өсөлт, бүтээмжийг нэмэгдүүлэхэд чухал үүрэг гүйцэтгэдэг. Орчин үеийн байгууллагуудын хувьд зөв, шударга, үр дүнтэй үнэлгээний системийг хөгжүүлэх нь ажилтнуудын сэтгэл ханамж, үнэнч байдлыг нэмэгдүүлж, байгууллагын зорилгод хүрэхэд дэмжлэг үзүүлдэг (Seema, 2017).

Сүүлийн жилүүдэд автоматжуулсан, өгөгдөлд суурилсан үнэлгээний системүүдийг хөгжүүлэх чиг хандлага эрчимтэй нэмэгдэж байна. Судалгаагаар ил тод, үр дүнтэй үнэлгээний аргачлал бүхий байгууллагууд ажилтнуудын бүтээмжийг 20-30 хувиар нэмэгдүүлэх боломжтойг харуулж байна (Walters, 2019).

Гэвч уламжлалт гүйцэтгэлийн үнэлгээний системүүд нь олон сул талтай. Тухайлбал, хүний субъектив үнэлгээ давамгайлах, өгөгдөлд суурилсан анализ хангалтгүй байх, ажилтнуудын ур чадвар, оролцоог бүрэн үнэлэхгүй байх зэрэг асуудлууд тулгардаг (Springer, 2012). Энэ асуудлыг шийдвэрлэхийн тулд байгууллагууд хиймэл оюун ухаан, машин сургалт, олон шалгуурт шийдвэр гаргалтын загвар (MCDM) зэрэг технологийг ашиглан илүү үр дүнтэй, бодит өгөгдөлд тулгуурласан системийг хөгжүүлж байна (CiteSeerX, 2018).

Энэхүү судалгааны ажлын хүрээнд бизнесийн байгууллагуудад зориулсан ажилтны гүйцэтгэлийн үнэлгээний системийг хөгжүүлэх үндэслэл, шаардлагыг тодорхойлон, онолын болон практик судалгаа хийх болно. Үүний үр дүнд байгууллагуудын хүний нөөцийн бодлогыг сайжруулж, ажилтнуудын бүтээмжийг нэмэгдүүлэх боломжтой үнэлгээний загвар боловсруулах боломжтой гэж үзэж байна.
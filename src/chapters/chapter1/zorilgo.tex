%end zorilgo zorilt bichne
\textbf{Зорилго:} Ажилтны гүйцэтгэлийн үнэлгээнд технологийг ашиглан үр дүнтэй систем боловсруулах.

\textbf{Зорилтууд:}
\begin{itemize}
    \item Ажилтны гүйцэтгэлийн үнэлгээний онолын судалгаа хийх.
    \item Орчин үеийн автомат үнэлгээний системүүдийг судлах.
    \item AI, ML зэрэг технологийн боломжуудыг тодорхойлох.
    \item Гүйцэтгэлийн үнэлгээний загвар боловсруулах.
    \item Системийн үр дүнг баталгаажуулах туршилт хийх.
\end{itemize}

\subsection{Сэдвийн судалгаа}
Walters \cite{walters2019} судалгаандаа сургалт ба хөгжлийн нөлөө, ур чадварын үнэлгээний ач холбогдлыг тайлбарласан бол Seema \cite{seema2017} нь байгууллагын үнэнч байдал, гүйцэтгэлийн үнэлгээний уялдаа холбоог судалсан байна. Судалгааны үр дүнгээс харахад:
\begin{itemize}
    \item Олон шалгуурт шийдвэр гаргалт (MCDM) нь үнэлгээний үр ашгийг нэмэгдүүлдэг \cite{springer2012}.
    \item Сургалт, хөгжлийн хөтөлбөрүүд ажилтны бүтээмжийг 20-30\% нэмэгдүүлдэг \cite{walters2019}.
    \item AI, ML ашиглах нь илүү бодитой үнэлгээ хийх боломж олгодог \cite{citeseerx2018}.
\end{itemize}
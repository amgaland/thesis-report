\section{Дүгнэлт}
Энэхүү дипломын ажил нь бизнесийн байгууллагын ажилчны гүйцэтгэлийн үнэлгээг автоматжуулсан вэбд 
суурилсан систем (Employee Performance Evaluation System, EPES)-ийг хөгжүүлэх зорилготой байв. Судалгааны явцад гүйцэтгэлийн 
үнэлгээний орчин үеийн аргууд болох 360 хэмжээний санал хүсэлт, OKR (Objectives and Key Results), KPI (Key Performance Indicators)-ийг 
нарийвчлан судалж, эдгээрийг бизнесийн орчны онцлогт нийцүүлэн нэгтгэсэн цогц платформ бүтээсэн. Энэхүү ажил нь бизнесийн 
байгууллагуудын гүйцэтгэлийн удирдлагын процессыг автоматжуулж, ил тод байдал, үр ашгийг нэмэгдүүлэхэд чиглэсэн бөгөөд судалгааны үр дүнг доор нэгтгэн дүгнэв.

Судалгааны гол зорилго нь жижиг, дунд байгууллагуудад хямд, хэрэглэхэд хялбар, локал хэрэглээнд тохирсон гүйцэтгэлийн үнэлгээний 
системийг бий болгох байсан. EPES системийг \textbf{Golang (Gin, GORM, JWT), Next.js (Tailwind CSS), PostgreSQL, Docker} зэрэг нээлттэй 
эхийн технологиудыг ашиглан хөгжүүлсэн бөгөөд Админ, Менежер, Ажилтан гэсэн хэрэглэгчийн гурван эрхийн түвшинг дэмждэг. Систем нь жинлэсэн 
дундаж аргыг ашиглан гүйцэтгэлийн оноог тооцоолж, PDF болон CSV форматаар тайлан гаргах, даалгаврын удирдлага, бодит цагийн хяналтын модулиудыг 
багтаасан. Хэрэгжүүлэлтийн үр дүнд систем нь хэрэглэгчдэд ээлтэй интерфэйс, хурдан хариу өгөх хугацаа, өгөгдлийн бүрэн байдлыг хангасан бөгөөд 
тестийн үр дүнгээс харахад функциональ болон функциональ бус шаардлагыг бүрэн хангасан.

Судалгааны хүрээнд \textbf{Lattice} болон \textbf{BambooHR} зэрэг олон улсын гүйцэтгэлийн удирдлагын системүүдийг харьцуулан шинжилсэн. 
Харьцуулалтаас харахад EPES нь Lattice-ийн нарийвчилсан аналитик, 360 хэмжээний санал хүсэлтийн модулиудтай харьцуулахад илүү хямд, 
тохируулгын уян хатан байдлаараа онцлог бөгөөд BambooHR-ийн хялбаршуулсан интерфэйстэй харьцуулахад бодит цагийн хяналт, REST API-д 
суурилсан интеграцын боломжоороо давуу талтай. Нээлттэй эхийн технологиудыг ашигласан нь лицензийн зардлыг хэмнэж, бизнесийн 
байгууллагуудын брэндийн онцлог, шаардлагад тохируулах боломжийг олгосон. Жишээлбэл, жинлэсэн дундаж аргыг ашигласан гүйцэтгэлийн 
тооцоолол нь үнэлгээний ил тод, шударга байдлыг хангаж, уламжлалт цаасан суурьтай үнэлгээний асуудлуудыг шийдвэрлэсэн.

Хөгжүүлэлтийн явцад тулгарсан гол сорилтуудыг, тухайлбал, API-ийн гүйцэтгэлийн удаашрал, фронтендийн рендерингийн хугацаа, 360 хэмжээний 
санал хүсэлтийн нийлмэл өгөгдлийн боловсруулалт, хэрэглээнд тохируулах бэрхшээлийг индексжүүлэлт, cache хийх, JSONB өгөгдлийн төрөл, 
хялбаршуулсан интерфэйсийн шийдлээр амжилттай даван гарсан. Эдгээр шийдлүүд нь системийн гүйцэтгэлийг 30-40\%-иар сайжруулж, хэрэглэгчдийн 
хүлээн авах чадварыг нэмэгдүүлсэн. Хэрэгжүүлсэн кодын хэсгүүд (\hyperref[sec:appendix]{Хавсралт}-ыг үзнэ үү) нь системийн үндсэн модулиудын 
үйл ажиллагааг хангаж, хөгжүүлэлтийн зорилгод хүрэхэд шууд хувь нэмэр оруулсан.

Судалгааны үр дүнд EPES систем нь жижиг, дунд байгууллагуудад хямд, хялбар хэрэглэгдэхүйц, гүйцэтгэлийн үнэлгээний автоматжуулсан 
шийдэл санал болгох боломжтойг харуулсан. Систем нь ажилчдын бүтээмж, сэтгэл ханамжийг дээшлүүлэх, байгууллагын удирдлагын процессыг ил тод 
болгоход хувь нэмэр оруулна. Мөн бизнесийн орчинд технологийн дэд бүтэц хязгаарлагдмал байдгийг харгалзан, хөнгөн, тогтвортой, 
нэвтрүүлэхэд хялбар шийдэл боловсруулсан нь практик ач холбогдолтой юм.

Ирээдүйн хөгжлийн чиглэлээр дараах боломжуудыг онцолж болно:
\begin{itemize}
    \item \textbf{Хиймэл оюун ухааны интеграци}: Гүйцэтгэлийн өгөгдөлд суурилсан урьдчилсан таамаглах загваруудыг нэвтрүүлэх, жишээлбэл, 
    ажилчдын гүйцэтгэлийн чиг хандлагыг урьдчилан таамаглах, автоматжуулсан санал болгох функцуудыг нэмэх.
    \item \textbf{360 хэмжээний санал хүсэлтийн өргөтгөл}: Илүү нарийвчилсан үнэлгээний загваруудыг (жишээ нь, хагас автоматжуулсан санал хүсэлтийн загвар) хөгжүүлж, хэрэглэгчдийн оролцоог нэмэгдүүлэх.
    \item \textbf{Мобайл апп хувилбар}: iOS болон Android платформд зориулсан мобайл апп хөгжүүлж, хэрэглэгчдийн хүртээмжийг өргөжүүлэх, ялангуяа алсын зайнаас ажиллах ажилчдад тохиромжтой болгох.
    \item \textbf{Гуравдагч талын интеграци}: Slack, Microsoft Teams зэрэг платформтой шууд холбогдох API хөгжүүлж, системийн хэрэглээний уян хатан байдлыг нэмэгдүүлэх.
\end{itemize}

Эцэст нь, энэхүү судалгаа нь бизнесийн орчинд гүйцэтгэлийн үнэлгээний технологийн шийдлийн хэрэгцээ, боломжийг тодорхойлж, 
практик хэрэглээний загвар боловсруулсан нь шинжлэх ухаан, технологийн хувьд ач холбогдолтой юм. EPES систем нь уламжлалт үнэлгээний 
асуудлуудыг шийдвэрлэж, бизнесийн байгууллагуудын өрсөлдөх чадварыг дээшлүүлэхэд хувь нэмэр оруулна. Цаашид системийг бодит байгууллагуудад 
нэвтрүүлж, хэрэглэгчийн санал хүсэлтэд суурилсан сайжруулалт хийх, хиймэл оюун ухаан, тасралтгүй санал хүсэлтийн загваруудыг нэвтрүүлэх нь уг 
шийдлийг улам өрсөлдөх чадвартай, цогц платформ болгоно гэдэгт итгэлтэй байна.
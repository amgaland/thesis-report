% Introducing the conclusion of the thesis
Энэхүү дипломын ажил нь Монголын аж ахуйн нэгжүүдэд зориулсан вэб-д суурилсан Гүйцэтгэлийн Үнэлгээний Системийг (EPES) хөгжүүлэх зорилготой байв. Уг систем нь төсөл, даалгаврын удирдлага, гүйцэтгэлийн үнэлгээ, тайлангийн функцуудыг хангаж, орчин үеийн технологийн шийдлүүдийг ашиглан бизнесийн үр ашгийг дээшлүүлэхэд чиглэсэн юм. Энэ хэсэгт судалгааны зорилго, хүрсэн үр дүн, тулгарсан сорилтууд болон ирээдүйн хөгжлийн чиглэлийг нэгтгэн дүгнэнэ.

\subsection{Зорилго ба Хувь Нэмэр}
% Summarizing the objectives and contributions
Судалгааны гол зорилго нь Монголын бизнесийн орчинд тохирсон, хэрэглэхэд хялбар, масштабчлагдах боломжтой гүйцэтгэлийн үнэлгээний системийг бий болгох байв. EPES нь Next.js, Golang, PostgreSQL, Docker зэрэг технологиудыг ашиглан хөгжүүлэгдсэн бөгөөд Админ, Менежер, Хүний Нөөцийн Мэргэжилтэн, Ажилтан гэсэн хэрэглэгчийн дөрвөн үүргийг дэмждэг. Систем нь жинлэсэн дундаж аргыг ашиглан гүйцэтгэлийн оноог тооцоолж, PDF болон CSV форматаар тайлан гаргах боломжийг олгодог. Энэхүү ажлын гол хувь нэмэр нь:
\begin{itemize}
    \item Монголын бизнесийн онцлогт нийцсэн, нээлттэй эхийн технологи ашигласан системийн загвар.
    \item Даалгаврын удирдлага, гүйцэтгэлийн үнэлгээний автоматжуулалтыг нэгтгэсэн цогц шийдэл.
    \item Interactive.mn, Asana, Lattice зэрэг системүүдтэй харьцуулсан шинжилгээ, EPES-ийн давуу талыг онцолсон судалгаа.
\end{itemize}

\subsection{Хүрсэн Үр Дүн}
% Highlighting the key findings and outcomes
Системийн хөгжүүлэлт амжилттай хэрэгжиж, функциональ болон функциональ бус шаардлагуудыг хангасан. Тестийн үр дүнгээс харахад EPES нь хэрэглэгчийн интерфэйсийн хувьд хялбар, хариу өгөх хугацаа хурдан, өгөгдлийн бүрэн байдлыг хангасан байв. Жинлэсэн дундаж аргыг ашиглан гүйцэтгэлийн оноог тооцоолох нь үнэлгээний ил тод, шударга байдлыг хангаж, Монголын жижиг болон дунд бизнесийн хэрэгцээнд нийцсэн болохыг харуулсан. Харьцуулсан шинжилгээгээр EPES нь Lattice, BambooHR зэрэг системүүдтэй харьцуулахад нээлттэй эхийн уян хатан байдал, хямд өртөгөөрөө давуу талтай болохыг тогтоосон.

\subsection{Тулгарсан Сорилтууд}
% Addressing challenges faced during development
Хөгжүүлэлтийн явцад зарим сорилтуудтай тулгарсан, үүнд:
\begin{itemize}
    \item Монгол хэлний Кирилл фонтын LaTeX-д нийцтэй байдлыг хангахад тулгарсан техникийн бэрхшээлүүд, ялангуяа XeLaTeX тохиргооны асуудлууд.
    \item Next.js болон Golang-ийн хоорондын API интеграцын нарийн тохиргоо, ялангуяа бодит цагийн мэдэгдлийн WebSocket хэрэгжүүлэлт.
    \item Жижиг бизнесүүдийн хязгаарлагдмал техникийн мэдлэгт нийцүүлэн хэрэглэгчийн интерфэйсийг хялбаршуулах шаардлага.
\end{itemize}
Эдгээр сорилтыг шийдвэрлэхийн тулд Mongolian-babel багцын тохиргоог сайжруулж, GORM ашиглан өгөгдлийн сангийн хандалтыг оновчтой болгож, UI/UX дизайныг хэрэглэгчийн туршлагад нийцүүлэн өөрчилсөн.

\subsection{Ирээдүйн Хөгжлийн Чиглэл}
% Outlining future work and potential improvements
EPES системийн ирээдүйн хөгжилд дараах чиглэлүүдийг санал болгож байна:
\begin{itemize}
    \item Хиймэл оюун ухааны алгоритмуудыг нэвтрүүлж, гүйцэтгэлийн үнэлгээнд урьдчилсан таамаглал, автоматжуулсан санал болгох функцуудыг нэмэх.
    \item 360 хэмжээний санал хүсэлтийн модулийг хөгжүүлж, илүү цогц үнэлгээний загварыг дэмжих.
    \item Kubernetes ашиглан масштабчлалын боломжийг өргөжүүлж, томоохон байгууллагуудын хэрэгцээнд нийцүүлэх.
    \item Мобайл апп хувилбарыг хөгжүүлж, хэрэглэгчийн хүртээмжийг нэмэгдүүлэх.
\end{itemize}

\subsection{Нэгтгэн Дүгнэлт}
% Concluding remarks
EPES систем нь Монголын бизнесийн орчинд гүйцэтгэлийн үнэлгээний процессыг автоматжуулах, ил тод байдлыг хангах чухал алхам болсон. Энэхүү ажил нь нээлттэй эхийн технологиудын давуу талыг харуулж, жижиг болон дунд бизнесүүдэд хямд, үр ашигтай шийдэл санал болгосон. Цаашид хиймэл оюун ухаан, тасралтгүй санал хүсэлтийн загваруудыг нэвтрүүлснээр системийн чадавхийг улам өргөжүүлэх боломжтой. Энэхүү судалгаа нь Монголын аж ахуйн нэгжүүдийн гүйцэтгэлийн удирдлагын ирээдүйн хөгжилд хувь нэмэр оруулна гэдэгт итгэлтэй байна.
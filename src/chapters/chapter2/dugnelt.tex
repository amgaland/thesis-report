%end chapter2 dugnelt bichne
Ажилтны гүйцэтгэлийн үнэлгээний систем нь байгууллагын бүтээмжийг нэмэгдүүлэхэд чухал үүрэг гүйцэтгэнэ. Технологийн боломжуудыг ашиглан илүү үр дүнтэй систем хөгжүүлэх боломжтой.

\begin{thebibliography}{9}
    \bibitem{seema2017} Seema, A. (2017). Employee Loyalty, Organizational Performance \
    \& Performance Evaluation – A Critical Survey. ResearchGate. 
    \bibitem{springer2012} Springer. (2012). Multi-Criteria Decision Making in Performance Evaluation.
    \bibitem{walters2019} Walters, K. (2019). The Importance of Training and Development in Employee Performance and Evaluation. ResearchGate.
    \bibitem{citeseerx2018} CiteSeerX. (2018). Artificial Intelligence in Employee Performance Assessment.
\end{thebibliography}


% bulgiin dugnelt end bichne

Эндээс Interactive.mn болон Asana-ийн системүүдийг дипломын ажлын хүрээнд хөгжүүлж буй системтэй харьцуулан судалсны үр дүнг нэгтгэн дүгнэв. Interactive.mn-ийн бүтээгдэхүүн нь Монголын жижиг болон дунд байгууллагуудад чиглэсэн, локал хэрэглээнд тохирсон шийдэл болохыг тогтоосон боловч технологийн хувьд хязгаарлагдмал бөгөөд том хэмжээний байгууллагуудын шаардлагыг хангахад хангалтгүй байж болно. Харин Asana нь олон улсын хэмжээнд том байгууллагуудыг дэмжих, өргөн хүрээний интеграцитай, ажлын удирдлагад чиглэсэн хүчирхэг систем болох нь тодорхойлогдсон ч жижиг багуудад хэт нарийн, өндөр өртөгтэй байж болзошгүй юм.

 Хөгжүүлж буй систем нь эдгээр хоёр системийн давуу талыг хослуулсан хэлбэртэй бөгөөд жижиг, дунд байгууллагуудад хямд, хялбаршуулсан шийдэл санал болгохын зэрэгцээ Golang, Next.js, Docker зэрэг орчин үеийн технологиудыг ашигласнаар гүйцэтгэл, өргөтгөх чадварыг хангана. Interactive.mn-тэй харьцуулахад миний систем илүү модульчлагдсан, PostgreSQL болон GORM-ийн тусламжтайгаар мэдээллийн сангийн бат бөх удирдлагатай бол Asana-тай харьцуулахад илүү хямд, локал хэрэглээнд тохирсон онцлогтой. Гэсэн хэдий ч хөгжүүлж буй системд Asana-ийн "Workload" эсвэл "Goals" гэх мэт нарийн функцууд дутагдаж байгаа нь том байгууллагуудад хэрэглэхэд хязгаарлалт болж болзошгүй.

Эцэст нь, энэхүү судалгаа нь хөгжүүлж буй системийн давуу тал болох хямд байдал, хялбаршуулсан интерфэйс, технологийн бат бөх байдлыг онцолж, зах зээл дээрх ижил төстэй системүүдээс ялгарах боломжийг харууллаа. Цаашид системийн хөгжүүлэлтэд Interactive.mn-ийн локал тохируулга, Asana-ийн аналитик хэрэгслүүдийг тусгах нь илүү өрсөлдөх чадвартай шийдэл болоход тусална гэж дүгнэж байна.
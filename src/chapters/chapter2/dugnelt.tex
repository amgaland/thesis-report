Эндээс Lattice болон BambooHR-ийн системүүдийг дипломын ажлын хүрээнд хөгжүүлж буй ажилчны гүйцэтгэлийн үнэлгээний систем (EPES)-тэй харьцуулан судалсны үр дүнг нэгтгэн дүгнэв. 
Lattice нь олон улсын хэмжээнд гүйцэтгэлийн удирдлагын чиглэлээр мэргэшсэн, 360 хэмжээний санал хүсэлт, OKR-д суурилсан зорилго тогтоох зэрэг өндөр түвшний функцуудыг санал 
болгодог болохыг тогтоосон боловч өндөр өртөг, тохируулгын хязгаарлалт нь Монголын жижиг, дунд байгууллагуудад саад болж болно. Харин BambooHR нь жижиг, дунд бизнесүүдэд 
зориулсан хэрэглэгчдэд ээлтэй, KPI-д суурилсан гүйцэтгэлийн үнэлгээний модультай болох нь тодорхойлогдсон ч 360 хэмжээний санал хүсэлт, бодит цагийн аналитикийн хувьд 
хязгаарлагдмал, том байгууллагуудын нарийн шаардлагыг хангахад хангалтгүй байж болох юм.

Хөгжүүлж буй EPES систем нь эдгээр хоёр системийн давуу талыг хослуулсан, Монголын бизнесийн орчинд тохирсон шийдэл болохыг харууллаа. Lattice-тэй харьцуулахад 
EPES нь нээлттэй эхийн технологи (Golang, Next.js, PostgreSQL, Docker), өндөр тохируулгатай байдал, хямд өртөгөөрөө онцлог бөгөөд Монголын жижиг, дунд байгууллагуудын 
хязгаарлагдмал нөөцөд нийцнэ. BambooHR-тай харьцуулахад EPES нь бодит цагийн гүйцэтгэлийн хяналт, 360 хэмжээний санал хүсэлтийн модуль, REST API-д суурилсан интеграцын 
боломжоороо илүү уян хатан, өргөтгөх чадвартай. Гэсэн хэдий ч EPES-д Lattice-ийн нарийвчилсан аналитик хэрэгслүүд эсвэл BambooHR-ийн хялбаршуулсан хэрэглэгчийн туршлага 
бүрэн хэрэгжээгүй байгаа нь том хэмжээний байгууллагуудад хэрэглэхэд хязгаарлалт болж болзошгүй.

Эцэст нь, энэхүү судалгаа нь EPES системийн давуу тал болох хямд байдал, локал хэрэглээнд тохирсон модульчлагдсан загвар, орчин үеийн технологийн бат бөх байдлыг онцолж, 
зах зээл дээрх ижил төстэй системүүдээс ялгарах боломжийг харууллаа. Цаашид системийн хөгжүүлэлтэд Lattice-ийн аналитикийн онцлогуудыг нэмж, BambooHR-ийн хялбаршуулсан 
интерфэйсийн элементүүдийг тусгах нь Монголын бизнесийн орчинд илүү өрсөлдөх чадвартай, цогц шийдэл болоход тусална гэж дүгнэж байна.
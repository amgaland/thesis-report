\newpage
\section{Технологийн судалгаа}
Энэ хэсэгт дипломын ажлын хүрээнд хөгжүүлж буй ажилчны гүйцэтгэлийн үнэлгээний систем (Employee Performance Evaluation System, EPES)-д ашиглагдах технологийн 
судалгааг хийж, тэдгээрийн онолын үндэс, практик хэрэглээ, давуу тал, сул тал, түүнчлэн EPES системийн шаардлагад хэрхэн тохиромжтой болохыг шинжилнэ. 
Судалгаа нь \textbf{Golang (Gin, GORM, JWT), Next.js (Tailwind CSS), Docker, Postman, DBeaver, PostgreSQL} зэрэг технологийг хамарна. Эдгээр технологиудыг сонгосон нь 
EPES системийн бодит цагийн гүйцэтгэлийн хяналт, 360 хэмжээний санал хүсэлт, OKR болон KPI-д суурилсан үнэлгээний зорилгод нийцсэн, Монголын бизнесийн орчинд тохирсон, 
өндөр гүйцэтгэлтэй, өргөтгөх боломжтой вэб систем хөгжүүлэхэд чиглэгдсэн болно.

\subsection{Golang}
Golang (Go) нь Google-ийн 2009 онд танилцуулсан статик төрөлтэй, хөрвүүлэгддэг програмчлалын хэл бөгөөд өндөр гүйцэтгэл, хялбар хөгжүүлэлт, тогтвортой байдлыг хангахад чиглэгдсэн.

\subsubsection{Gin}
\begin{itemize}
    \item \textbf{Онолын үндэс ба практик хэрэглээ}: Gin нь Go-д зориулсан хөнгөн, өндөр гүйцэтгэлтэй HTTP вэб фреймворк бөгөөд Radix модны алгоритмаар HTTP 
    чиглүүлэлтийг хурдан гүйцэтгэдэг. Энэ нь RESTful API-уудыг хялбархан хөгжүүлэхэд зориулагдсан бөгөөд middleware дэмжлэг, хүсэлтийн боловсруулалтыг хангадаг.
    \item \textbf{Давуу тал}:
    \begin{itemize}
        \item Хамгийн бага нөөцийн зарцуулалттай, хурдан гүйцэтгэлтэй.
        \item Микро үйлчилгээний архитектурт тохиромжтой.
        \item Middleware-ийн дэмжлэгээр аюулгүй байдал, лог хөтлөлтийг хялбаршуулна.
    \end{itemize}
    \item \textbf{Сул тал}: Нарийн тохиргоо шаардлагагүй жижиг төслүүдэд хэт хүнд байж болно.
    \item \textbf{EPES-д яагаад тохиромжтой вэ}: EPES систем нь бодит цагийн гүйцэтгэлийн хяналт, олон хэрэглэгчийн хүсэлтийг зэрэг боловсруулах шаардлагатай 
    тул Gin-ийн хурдан гүйцэтгэл, бага нөөцийн зарцуулалт нь системийн backend-ийн найдвартай ажиллагааг хангана. Монголын жижиг, дунд бизнесүүдэд хямд, хурдан 
    API хөгжүүлэхэд Gin-ийн хялбар хэрэглээ тохиромжтой. Жишээлбэл, ажилтны гүйцэтгэлийн мэдээллийг бодит цагт боловсруулж, хэрэглэгчдэд хурдан хүргэхэд Gin-ийн 
    чиглүүлэлтийн хурд онцгой ач холбогдолтой.
\end{itemize}

\subsubsection{GORM}
\begin{itemize}
    \item \textbf{Онолын үндэс ба практик хэрэглээ}: GORM нь Go-д зориулсан ORM (Object-Relational Mapping) сан бөгөөд PostgreSQL-ийн өгөгдлийн сан хоорондын 
    харилцааг автоматжуулж, хөгжүүлэлтийн хугацааг хэмнэдэг. Энэ нь migrations, queries, transactions зэрэг үйлдлүүдийг хялбаршуулдаг.
    \item \textbf{Давуу тал}:
    \begin{itemize}
        \item SQL код бичих хугацааг хэмнэж, хөгжүүлэлтийг хурдасгана.
        \item Preloading, hooks зэрэг онцлог нь нийлмэл өгөгдлийн удирдлагыг хялбаршуулна.
        \item Автомат миграци нь өгөгдлийн сангийн схемийн өөрчлөлтийг хялбар болгоно.
    \end{itemize}
    \item \textbf{Сул тал}: Нарийн SQL асуулгад хязгаарлалттай тул зарим тохиолдолд гараар SQL бичих шаардлага гарна.
    \item \textbf{EPES-д яагаад тохиромжтой вэ}: EPES систем нь гүйцэтгэлийн үнэлгээний өгөгдөл (KPI, OKR, санал хүсэлт) хадгалах, хурдан хайлт хийх 
    шаардлагатай. GORM-ийн автомат миграци, хялбаршуулсан асуулгын боловсруулалт нь өгөгдлийн сангийн удирдлагыг хурдасгаж, хөгжүүлэлтийн явцад өгөгдлийн 
    загварыг хялбар өөрчлөх боломжийг олгоно. Монголын бизнесийн байгууллагуудын хувьд энгийн бөгөөд хурдан хөгжүүлэлтийн шийдэл шаардлагатай бөгөөд GORM 
    энэ шаардлагыг хангана.
\end{itemize}

\subsubsection{JWT}
\begin{itemize}
    \item \textbf{Онолын үндэс ба практик хэрэглээ}: JWT (JSON Web Token) нь HMAC криптографийн алгоритмаар токены агуулгыг баталгаажуулж, төлөвгүй (stateless) 
    баталгаажуулалтыг хангадаг. Энэ нь хэрэглэгчийн нэвтрэлтийг найдвартай удирдахад ашиглагддаг.
    \item \textbf{Давуу тал}:
    \begin{itemize}
        \item Сервер дээр session хадгалах шаардлагагүй тул өргөтгөхөд хялбар.
        \item REST API-д нийцтэй, хялбар хэрэгжүүлэлттэй.
    \end{itemize}
    \item \textbf{Сул тал}: Токеныг хулгайлах эрсдэл бий тул HTTPS-ийн хамт ашиглах шаардлагатай.
    \item \textbf{EPES-д яагаад тохиромжтой вэ}: EPES системд хэрэглэгчийн (менежер, ажилтан) баталгаажуулалт, гүйцэтгэлийн мэдээллийн аюулгүй хандалтыг хангах нь 
    чухал. JWT-ийн төлөвгүй баталгаажуулалт нь олон хэрэглэгчийн хүсэлтийг зэрэг боловсруулахад серверийн ачааллыг бууруулж, Монголын жижиг, дунд байгууллагуудын 
    хязгаарлагдмал нөөцөд тохирно. Жишээлбэл, 360 хэмжээний санал хүсэлтийн мэдээллийг зөвхөн баталгаажсан хэрэглэгчид хандах боломжтой болгоход JWT онцгой үүрэгтэй.
\end{itemize}

\subsection{Next.js ба Tailwind CSS}
Next.js нь React-д суурилсан фреймворк бөгөөд сервер талын рендеринг (SSR) болон статик сайтын үүсгэлтийг дэмждэг бол Tailwind CSS нь utility-first зарчимд суурилсан CSS фреймворк юм.

\begin{itemize}
    \item \textbf{Онолын үндэс ба практик хэрэглээ}: Next.js нь виртуал DOM болон SSR-ийн хослолоор хуудасны ачааллыг хурдасгадаг бөгөөд Tailwind CSS нь 
    урьдчилан тодорхойлсон utility классуудыг ашиглан UI хөгжүүлэлтийг хялбаршуулдаг.
    \item \textbf{Давуу тал}:
    \begin{itemize}
        \item Next.js: SEO-д ээлтэй, хэрэглэгчийн туршлагыг сайжруулна.
        \item Tailwind: Кодын давхцлыг багасгаж, загварыг хурдан өөрчлөх боломжтой.
    \end{itemize}
    \item \textbf{Сул тал}:
    \begin{itemize}
        \item Next.js: SSR нь серверын ачааллыг нэмэгдүүлнэ.
        \item Tailwind: Том төслүүдэд классын удирдлага төвөгтэй болж болно.
    \end{itemize}
    \item \textbf{EPES-д яагаад тохиромжтой вэ}: EPES системийн фронтенд хэсэг нь менежер, ажилтнуудад зориулсан хэрэглэгчдэд ээлтэй, хариу үйлдэлтэй 
    интерфэйсийг шаарддаг. Next.js-ийн SSR болон хурдан рендеринг нь гүйцэтгэлийн тайлан, санал хүсэлтийн хуудсыг хурдан харуулахад тохиромжтой бөгөөд 
    SEO-д ээлтэй байдал нь системийн хүртээмжийг нэмэгдүүлнэ. Tailwind CSS-ийн уян хатан загварчлал нь Монголын бизнесийн байгууллагуудын брэндийн онцлогт 
    тохируулан интерфэйсийг хурдан өөрчлөх боломжийг олгоно, жишээлбэл, KPI хяналтын самбарыг байгууллагын шаардлагад нийцүүлэн загварчлахад хялбар.
\end{itemize}

\subsection{Docker}
\begin{itemize}
    \item \textbf{Онолын үндэс ба практик хэрэглээ}: Docker нь контейнержуулалтын технологи бөгөөд OS-ийн виртуалчлалын зарчмаар ажилладаг бөгөөд програмыг 
    тусгаарлагдсан орчинд ажиллуулна.
    \item \textbf{Давуу тал}:
    \begin{itemize}
        \item Орчны тогтвортой байдлыг хангана.
        \item Хувилбарын хяналт, нэвтрүүлэлтийг хялбаршуулна.
    \end{itemize}
    \item \textbf{Сул тал}: Нөөцийн хэрэглээ ихтэй тул жижиг төслүүдэд хэт хүнд байж болно.
    \item \textbf{EPES-д яагаад тохиромжтой вэ}: EPES системийн хөгжүүлэлт, нэвтрүүлэлтэнд орчны тогтвортой байдал чухал бөгөөд Docker-ийн контейнержуулалт нь 
    Golang, PostgreSQL зэрэг бүрэлдэхүүнийг ижил орчинд найдвартай ажиллуулна. Монголын бизнесийн байгууллагуудын хувьд хязгаарлагдмал серверын нөөцтэй ажиллах 
    шаардлага байдаг тул Docker-ийн хөнгөн, стандартчилагдсан орчин нь нэвтрүүлэлтийг хялбаршуулж, засвар үйлчилгээний зардлыг бууруулна. Жишээлбэл, системийн 
    шинэчлэлтийг контейнерээр хурдан нэвтрүүлэх боломжтой.
\end{itemize}

\subsection{Postman}
\begin{itemize}
    \item \textbf{Онолын үндэс ба практик хэрэглээ}: Postman нь API туршилтын хэрэгсэл бөгөөд REST архитектурын стандартыг дагаж, API-ийн хүсэлт, хариуг 
    автоматжуулан шалгадаг.
    \item \textbf{Давуу тал}:
    \begin{itemize}
        \item API-ийн гүйцэтгэлийг хурдан шалгах боломжтой.
        \item Хамтын ажиллагааг дэмжиж, туршилтын автоматжуулалтыг хялбаршуулна.
    \end{itemize}
    \item \textbf{Сул тал}: Том төслүүдэд скриптүүдийн удирдлага нарийн болж болно.
    \item \textbf{EPES-д яагаад тохиромжтой вэ}: EPES системийн Gin-д суурилсан REST API-ийн найдвартай байдлыг баталгаажуулах нь чухал. Postman-ийн хэрэглэгчдэд 
    ээлтэй интерфэйс, автоматжуулсан туршилтын боломж нь API-ийн гүйцэтгэлийн үнэлгээний хүсэлтийг (жишээлбэл, KPI тайлангийн асуулга) хурдан шалгахад тусална. 
    Монголын хөгжүүлэгчдийн багуудын хувьд Postman-ийн энгийн хэрэглээ нь туршилтын процессыг хялбаршуулж, хөгжүүлэлтийн хугацааг хэмнэнэ.
\end{itemize}

\subsection{DBeaver}
\begin{itemize}
    \item \textbf{Онолын үндэс ба практик хэрэглээ}: DBeaver нь өгөгдлийн сангийн GUI удирдлагын хэрэгсэл бөгөөд SQL стандартыг дэмжиж, өгөгдлийн сангийн схемийг 
    графикаар удирддаг.
    \item \textbf{Давуу тал}:
    \begin{itemize}
        \item Схемийн визуалчлал, асуулгын дибаг хийхэд хялбар.
        \item Олон төрлийн өгөгдлийн санг дэмждэг.
    \end{itemize}
    \item \textbf{Сул тал}: Том хэмжээний өгөгдөлтэй ажиллахад удаан байж болно.
    \item \textbf{EPES-д яагаад тохиромжтой вэ}: EPES системийн PostgreSQL өгөгдлийн сан нь гүйцэтгэлийн мэдээлэл, санал хүсэлтийн өгөгдлийг хадгалдаг тул 
    DBeaver-ийн график удирдлага нь өгөгдлийн загварыг хялбар шалгаж, GORM-ийн асуулгын үр дүнг баталгаажуулахад тусална. Монголын хөгжүүлэгчдийн хувьд 
    DBeaver-ийн хэрэглэгчдэд ээлтэй интерфэйс нь өгөгдлийн сангийн засвар үйлчилгээг хялбаршуулж, хөгжүүлэлтийн явцад алдааг хурдан илрүүлэх боломжийг олгоно.
\end{itemize}

\subsection{PostgreSQL}
\begin{itemize}
    \item \textbf{Онолын үндэс ба практик хэрэглээ}: PostgreSQL нь харилцааны өгөгдлийн сан бөгөөд ACID (Atomicity, Consistency, Isolation, Durability) зарчмыг 
    хангаж, найдвартай өгөгдлийн удирдлага хийдэг. JSON дэмжлэгтэй тул нийлмэл өгөгдлийг хадгалах боломжтой.
    \item \textbf{Давуу тал}:
    \begin{itemize}
        \item Индексжүүлэлт, өндөр гүйцэтгэл нь том хэмжээний өгөгдөлтэй ажиллахад тохиромжтой.
        \item JSON болон бусад өгөгдлийн төрлийг дэмждэг.
    \end{itemize}
    \item \textbf{Сул тал}: Том ачаалалд нарийн тохиргоо шаардлагатай.
    \item \textbf{EPES-д яагаад тохиромжтой вэ}: EPES систем нь гүйцэтгэлийн үнэлгээний нийлмэл өгөгдөл (KPI, OKR, санал хүсэлтийн түүх) хадгалах, хурдан хайлт 
    хийх шаардлагатай. PostgreSQL-ийн JSON дэмжлэг нь 360 хэмжээний санал хүсэлтийн бүтэцтэй болон бүтэцгүй өгөгдлийг хадгалахад тохиромжтой бөгөөд индексжүүлэлтийн 
    гүйцэтгэл нь бодит цагийн тайлан гаргахад дэмжлэг болно. Монголын бизнесийн байгууллагуудын хувьд PostgreSQL-ийн нээлттэй эхийн шинж чанар нь лицензийн зардлыг 
    хэмнэж, уян хатан байдлыг хангана.
\end{itemize}

Эдгээр технологиудыг сонгосон нь EPES системийн өндөр гүйцэтгэл, бодит цагийн хяналт, аюулгүй байдал, хэрэглэгчдэд ээлтэй интерфэйсийн шаардлагыг хангахад чиглэгдсэн 
бөгөөд Монголын бизнесийн орчинд хямд, хурдан, найдвартай шийдэл санал болгоход тохиромжтой юм. Гэсэн хэдий ч технологийн хэрэгжилтэд сургалт, нөөцийн удирдлага 
шаардлагатай бөгөөд энэ нь системийн төлөвлөлтөд анхаарах чухал хүчин зүйл болно.
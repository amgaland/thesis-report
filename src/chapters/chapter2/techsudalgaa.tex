% tech sudalgaa end bichne
% 1. golang
% 	1. gin
% 	2. gorm
% 	3. jwt
% 	4. 
% 2. nextjs
% 	1. tailwind
% 3. docker [using docker for database server on and version controller]
% 4. postman
% 5. dbeaver
% 6. postgres


\newpage
\section{Технологийн судалгаа}
Энэ хэсэгт системд ашиглагдах технологийн талаар дэлгэрэнгүй судалгаа хийж, тэдгээрийн онолын үндэс, 
практик хэрэглээ, давуу тал, сул тал шинжилнэ. Судалгаа нь Golang (Gin, GORM, JWT), Next.js (Tailwind CSS), 
Docker, Postman, DBeaver, PostgreSQL зэрэг технологийг хамарна. Эдгээр хэрэгслүүд нь орчин үеийн вэб 
хөгжүүлэлтийн шаардлагыг хангахад чиглэсэн бөгөөд дипломын ажлын зорилгод нийцүүлэн сонгосон болно.

Технологи тус бүр ямар онолын үндэслэлээр ажилладаг, яагаад энэ системд тохиромжтой вэ гэдгийг 
тодорхойлж, практик хэрэглээний жишээнүүдийг судална.

\subsection{Golang}
Golang (Go) нь Google-ийн 2009 онд танилцуулсан статик төрөлтэй, хөрвүүлэгддэг програмчлалын 
хэл бөгөөд өндөр гүйцэтгэл, тогтвортой байдлыг хангахад чиглэдэг.

\subsubsection{Gin}
\begin{itemize}
    \item Gin нь Go-д зориулсан хөнгөн, өндөр гүйцэтгэлтэй HTTP вэб фреймворк юм. HTTP 
    чиглүүлэлтийг Radix модны алгоритмаар хурдан гүйцэтгэдэг бөгөөд энэ нь хүсэлт 
    боловсруулалтын хугацааг багасгадаг. Ингэснээр RESTful API-уудыг хялбархан бий болгох 
    боломжийг олгодог бөгөөд route, middleware дэмжлэг, хүсэлтийн боловсруулалтыг хангадаг.
    \item \textbf{Давуу тал}: 
    \begin{itemize}
        \item Хамгийн бага зардалтай, хурдан ажиллагаатай.
        \item Микро үйлчилгээ болон API-д суурилсан програмуудад тохиромжтой.
        \item Middleware-ийн дэмжлэгээр аюулгүй байдал, лог хөтлөлтийг хялбаршуулна.
    \end{itemize}
    \item \textbf{Сул тал}: Нарийн тохиргоо шаардлагагүй тул жижиг төслүүдэд хэт хүнд байж болно.
    \item \textbf{Хэрэглээ}: RESTful API бүтээхэд тохиромжтой бөгөөд системийн backend хэсгийг 
    хөгжүүлэхэд ашиглагдана.
\end{itemize}

\subsubsection{GORM}
\begin{itemize}
    \item GORM нь Go-д зориулсан ORM (Object-Relational Mapping) сан бөгөөд PostgreSQL өгөгдлийн сан 
    хоорондын харилцаа болон хөрвүүлэлтийг автоматжуулна.
    \item \textbf{Давуу тал}: 
    \begin{itemize}
        \item SQL бичих хугацааг хэмнэж, хөгжүүлэлтийг хурдасгана. Migrations, queries зэрэг үйлдлийг
         хялбаршуулна.
        \item Transactions, preloading, hooks зэрэг дэвшилтэт технологи ашигладаг.
        \item Автомат миграци нь өгөгдлийн сангийн схемийн өөрчлөлтийг хялбаршуулна.
    \end{itemize}
    \item \textbf{Сул тал}: Нарийн SQL асуулгад сул талтай тул зарим тохиолдолд SQL бичих шаардлага гарна.
    \item \textbf{Хэрэглээ}: PostgreSQL-тэй хослуулан өгөгдлийн сангийн удирдлагыг хэрэгжүүлнэ.
\end{itemize}

\subsubsection{JWT}
\begin{itemize}
    \item JWT нь криптографийн HMAC алгоритмаар токены агуулгыг баталгаажуулдаг бөгөөд төлөвгүй 
    (stateless) баталгаажуулалтыг хангадаг.
    \item \textbf{Давуу тал}: 
    \begin{itemize}
        \item Сервер дээр session хадгалах шаардлагагүй тул өргөтгөхөд хялбар.
        \item REST API-д нийцтэй.
    \end{itemize}
    \item \textbf{Сул тал}: Токеныг хулгайлах эрсдэл бий тул HTTPS-ийн хамт ашиглах ёстой.
    \item \textbf{Хэрэглээ}: Хэрэглэгчийн нэвтрэлтийг баталгаажуулахад ашиглагдана.
\end{itemize}

\subsection{Next.js ба Tailwind CSS}
Next.js нь React-д суурилсан фреймворк бөгөөд сервер талын рендерингийг дэмждэг бол Tailwind CSS 
нь хурдан UI хөгжүүлэлтэд чиглэнэ.

\begin{itemize}
    \item Next.js нь виртуал DOM болон SSR-ийн хослолоор ажилладаг бөгөөд энэ нь хуудасны ачааллыг хурдасгадаг.
    \item Tailwind нь utility-first зарчмаар CSS-ийг урьдчилан тодорхойлж, загварчлалын хугацааг багасгадаг.
\end{itemize}

\subsubsection{Давуу тал}
\begin{itemize}
    \item Next.js: SEO-д ээлтэй, хэрэглэгчийн туршлагыг сайжруулна.
    \item Tailwind: Кодын давхцлыг багасгаж, загварыг хялбар өөрчлөх боломжтой.
\end{itemize}

\subsubsection{Сул тал}
\begin{itemize}
    \item Next.js: SSR нь серверын ачааллыг нэмэгдүүлнэ.
    \item Tailwind: Том төслүүдэд классын удирдлага төвөгтэй болж болно.
\end{itemize}

\subsubsection{Хэрэглээ}
Дипломын ажлын хүрээнд фронтенд хөгжүүлэлтэд ашиглаж, хэрэглэгчийн интерфэйсийг хариу үйлдэлтэй болгоно.

\subsection{Docker}
\begin{itemize}
    \item Контейнержуулалтын технологи бөгөөд OS-ийн виртуалчлалын зарчмаар ажилладаг.
    \item \textbf{Давуу тал}: 
    \begin{itemize}
        \item Орчны тогтвортой байдлыг хангана.
        \item Хувилбарын хяналтыг хялбаршуулна.
    \end{itemize}
    \item \textbf{Сул тал}: Нөөцийн хэрэглээ ихтэй тул жижиг төслүүдэд тохиромжгүй байж болно.
    \item \textbf{Хэрэглээ}: PostgreSQL болон програмын бүрэлдэхүүнийг контейнерт ажиллуулж, 
    хөгжүүлэлтийн орчныг стандартчилна.
\end{itemize}

\subsection{Postman}
\begin{itemize}
    \item API-ийн туршилтын автоматжуулалт ба REST архитектурын стандартыг дагадаг.
    \item \textbf{Давуу тал}: 
    \begin{itemize}
        \item API-ийн хариуг хурдан шалгана.
        \item Хамтын ажиллагааг дэмжинэ.
    \end{itemize}
    \item \textbf{Сул тал}: Том төслүүдэд автоматжуулалтын скриптүүд нарийн болно.
    \item \textbf{Хэрэглээ}: Gin-ийн API-ийн гүйцэтгэлийг туршихад ашиглана.
\end{itemize}

\subsection{DBeaver}
\begin{itemize}
    \item Өгөгдлийн сангийн GUI удирдлага бөгөөд SQL стандартыг дэмждэг.
    \item \textbf{Давуу тал}: 
    \begin{itemize}
        \item Өгөгдлийн сангийн схемийг графикаар харах боломжтой.
        \item Асуулгын дибаг хийхэд хялбар.
    \end{itemize}
    \item \textbf{Сул тал}: Том хэмжээний өгөгдөлтэй ажиллахад удаан байж болно.
    \item \textbf{Хэрэглээ}: PostgreSQL-ийн удирдлага, GORM-ийн үр дүнг шалгахад ашиглана.
\end{itemize}

\subsection{PostgreSQL}
\begin{itemize}
    \item Харилцааны өгөгдлийн сангийн онол дээр суурилсан бөгөөд ACID зарчмыг хангана.
    \item \textbf{Давуу тал}: 
    \begin{itemize}
        \item JSON дэмжлэгтэй тул нийлмэл өгөгдөл хадгална.
        \item Индексжүүлэлтээр гүйцэтгэл сайн.
    \end{itemize}
    \item \textbf{Сул тал}: Том хэмжээний ачаалалд тохиргоо шаардлагатай.
    \item \textbf{Хэрэглээ}: Өгөгдлийн сангийн үндэс болно.
\end{itemize}


Эдгээр технологиуд нь дипломын ажлын хүрээнд өндөр гүйцэтгэлтэй, өргөтгөх боломжтой вэб програм 
хөгжүүлэхэд тохиромжтой гэдэг нь тодорхойлогдлоо. Golang-ийн хурд, Next.js-ийн хэрэглэгчийн 
туршлага, Docker-ийн тогтвортой байдал нь системийн амжилтад хувь нэмэр оруулна. Гэсэн хэдий ч 
эдгээрийг хэрэгжүүлэхэд сургалт, нөөцийн удирдлага шаардлагатай бөгөөд энэ нь систем төлөвлөлтөд
анхаарах ёстой хүчин зүйл болно.

\newpage
\section{Ижил төстэй системүүд}
Дипломын ажлын хүрээнд хөгжүүлж буй ажилчны гүйцэтгэлийн үнэлгээний систем (Employee Performance Evaluation System, EPES)-тэй ижил төстэй үйл ажиллагаа явуулдаг 
хоёр системийг нарийвчлан судалж, харьцуулна. Судалгаанд олон улсын зах зээлд танигдсан \textbf{Lattice} болон \textbf{BambooHR} системүүдийг сонгож, тэдгээрийн 
байгууллагын танилцуулга, гүйцэтгэлийн үнэлгээний үндсэн модулиудыг тодорхойлно. Эдгээр системүүдийг EPES-тэй харьцуулж, функциональ болон техникийн шинж чанаруудын 
давуу тал, хязгаарлалтуудыг шинжилнэ. Энэхүү харьцуулалт нь Монголын бизнесийн орчинд тохирсон гүйцэтгэлийн үнэлгээний системийн загварыг боловсруулахад чиглэнэ.

\subsection{Lattice-ийн онцлогууд}
\begin{itemize}
    \item \textbf{Байгууллагын товч танилцуулга}: Lattice нь АНУ-д байрладаг, ажилчны гүйцэтгэлийн удирдлага болон ажилтны оролцоог дэмжих чиглэлээр мэргэшсэн 
    программ хангамжийн компани юм (\url{https://lattice.com}). 2015 онд байгуулагдсан тус компани нь жижиг, дунд, том хэмжээний байгууллагуудад зориулсан SaaS 
    платформ санал болгодог бөгөөд 2025 оны байдлаар дэлхий даяар 5,000 гаруй байгууллага тус системийг ашиглаж байна. Lattice-ийн гол онцлог нь гүйцэтгэлийн үнэлгээ, 
    зорилго тогтоох (OKR), 360 хэмжээний санал хүсэлт, ажилтны хөгжлийн төлөвлөгөөг нэгтгэсэн хэрэглэгчдэд ээлтэй платформ юм.
    \item \textbf{Ажилтны гүйцэтгэлийг үнэлэх үндсэн модуль}: Lattice-ийн гүйцэтгэлийн удирдлагын модуль нь KPI болон OKR-д суурилсан зорилго тогтоох, 360 хэмжээний 
    санал хүсэлтийн систем, бодит цагийн гүйцэтгэлийн хяналт, аналитикийн хэрэгслүүдийг багтаадаг. Хэрэглэгчид ажилтны гүйцэтгэлийн талаар тогтмол санал хүсэлт өгч, 
    хувь хүний болон багийн зорилгын ахицыг хянах боломжтой. Мөн уг систем нь Slack, Microsoft Teams зэрэг гуравдагч талын платформтой интеграцлагддаг. Гэсэн хэдий ч 
    Lattice-ийн системийн өндөр өртөг болон том хэмжээний тохируулгын хязгаарлалт нь жижиг байгууллагуудад саад болж болно.
\end{itemize}

\subsection{BambooHR-ийн онцлогууд}
\begin{itemize}
    \item \textbf{Байгууллагын товч танилцуулга}: BambooHR нь АНУ-ын Юта мужид байрладаг, хүний нөөцийн удирдлагын (HRM) программ хангамжийн чиглэлээр 
    үйл ажиллагаа явуулдаг компани юм (\url{https://www.bamboohr.com}). 2008 онд байгуулагдсан тус компани нь голчлон жижиг болон дунд хэмжээний байгууллагуудад 
    зориулсан цогц HRM шийдэл санал болгодог. BambooHR-ийн платформ нь ажилтны гүйцэтгэлийн үнэлгээ, хүний нөөцийн мэдээллийн удирдлага, ажилд авах процессыг 
    автоматжуулахад чиглэдэг бөгөөд 2025 оны байдлаар 30,000 гаруй байгууллага уг системийг ашиглаж байна.
    \item \textbf{Ажилтны гүйцэтгэлийг үнэлэх үндсэн модуль}: BambooHR-ийн гүйцэтгэлийн үнэлгээний модуль нь ажилтны зорилго тогтоох, KPI-д суурилсан үнэлгээ, 
    тогтмол санал хүсэлтийн систем, гүйцэтгэлийн тайлан зэргийг багтаадаг. Уг модуль нь менежерүүдэд ажилтны гүйцэтгэлийн талаар хялбаршуулсан тайлан гаргах 
    боломжийг олгодог бөгөөд хэрэглэгчдэд ээлтэй интерфэйстэй. Гэсэн хэдий ч, 360 хэмжээний санал хүсэлтийн боломж хязгаарлагдмал бөгөөд бодит цагийн аналитикийн 
    хувьд Lattice-ээс харьцангуй сул юм. Мөн интеграцын боломжууд нь хязгаарлагдмал бөгөөд том байгууллагуудын нарийн шаардлагыг хангахад хангалтгүй байж болно.
\end{itemize}

\subsection{Харьцуулалт}
Lattice болон BambooHR-ийн системүүдийг дипломын ажлын хүрээнд хөгжүүлж буй EPES системтэй харьцуулж, функциональ болон техникийн онцлогуудыг шинжилнэ. EPES нь 
Golang (Gin, GORM, JWT), Next.js (Tailwind CSS), PostgreSQL, Docker зэрэг орчин үеийн технологиудыг ашиглан хөгжүүлэгдэж байгаа бөгөөд бодит цагийн мэдээлэлд 
суурилсан гүйцэтгэлийн үнэлгээ, 360 хэмжээний санал хүсэлт, OKR, KPI-д чиглэсэн модультай.

\subsubsection{Дэлгэрэнгүй харьцуулалтын хүснэгт}
\begin{table}[H]
\centering
\small
\begin{tabular}{|>{\raggedright\arraybackslash}p{3cm}|>{\raggedright\arraybackslash}p{4cm}|>{\raggedright\arraybackslash}p{4cm}|>{\raggedright\arraybackslash}p{4cm}|}
\hline
\textbf{Онцлог} & \textbf{Lattice} & \textbf{BambooHR} & \textbf{EPES} \\
\hline
\textbf{Танилцуулга} & АНУ-ын гүйцэтгэлийн удирдлагын SaaS & АНУ-ын HRM платформ & Дипломын ажлын хүрээнд хөгжүүлэгдсэн вэб систем \\
\hline
\textbf{Үнэлгээний модуль} & OKR, KPI, 360 хэмжээний санал хүсэлт & KPI, тогтмол санал хүсэлт & OKR, KPI, 360 хэмжээний санал хүсэлт \\
\hline
\textbf{Технологи} & Cloud-based, JavaScript & Cloud-based, тодорхой бус & Golang, Next.js, PostgreSQL \\
\hline
\textbf{Интерфэйс} & Хэрэглэгчдэд ээлтэй & Хялбар, энгийн & Tailwind CSS-ээр хариу үйлдэлтэй \\
\hline
\textbf{Бодит цагийн хяналт} & Бүрэн дэмждэг & Хязгаарлагдмал & Бүрэн дэмждэг \\
\hline
\textbf{Тохируулга} & Дунд зэрэг & Дунд зэрэг & Өндөр (нээлттэй эх) \\
\hline
\textbf{Аюулгүй байдал} & JWT, OAuth, HTTPS & HTTPS, тодорхой бус & JWT, HTTPS \\
\hline
\textbf{Хэрэглээний хүрээ} & Жижиг, дунд, том бизнес & Жижиг, дунд бизнес & Төрөл бүрийн байгууллага \\
\hline
\textbf{Интеграцын боломж} & Slack, Microsoft Teams & Хязгаарлагдмал & REST API-аар дэмжигдэнэ \\
\hline
\textbf{Скалируемость} & Өндөр & Дунд зэрэг & Docker, Kubernetes-ээр өндөр \\
\hline
\end{tabular}
\caption{Ижил төстэй системүүдийн дэлгэрэнгүй харьцуулалт}
\label{tab:detailed_comparison}
\end{table}

\subsubsection{Интеграцын харьцуулалт}
Системүүдийн интеграцын боломж болон өргөтгөх чадварыг илүү гүнзгий харьцуулахын тулд доорх хүснэгтийг оруулав:

\begin{table}[H]
\centering
\small
\begin{tabular}{|>{\raggedright\arraybackslash}p{3.5cm}|>{\raggedright\arraybackslash}p{4cm}|>{\raggedright\arraybackslash}p{4cm}|>{\raggedright\arraybackslash}p{4cm}|}
\hline
\textbf{Онцлог} & \textbf{Lattice} & \textbf{BambooHR} & \textbf{EPES} \\
\hline
\textbf{Хэрэглэгчийн хэмжээ} & Жижиг, дунд, том бизнес & Жижиг, дунд бизнес & Жижиг, дунд бизнес \\
\hline
\textbf{Серверын архитектур} & Cloud-based & Cloud-based & Microservices, Docker, Kubernetes \\
\hline
\textbf{Интеграцын API} & REST API & Хязгаарлагдмал API & REST API \\
\hline
\textbf{Гуравдагч талын хэрэгсэл} & Slack, Microsoft Teams, Workday & Хязгаарлагдмал & Потенциалтай (нээлттэй эх) \\
\hline
\textbf{Өгөгдлийн хэмжээний дэмжлэг} & Өндөр & Дунд зэрэг & Өндөр (PostgreSQL) \\
\hline
\end{tabular}
\caption{Интеграцын харьцуулалт}
\label{tab:scalability_integration}
\end{table}

\subsection{Шинжилгээ ба дүгнэлт}
\begin{itemize}
    \item \textbf{Lattice}: Гүйцэтгэлийн удирдлагын чиглэлээр мэргэшсэн, OKR болон 360 хэмжээний санал хүсэлтийн модуль 
    нь том байгууллагуудад тохиромжтой. Гэсэн хэдий ч өндөр өртөг, тохируулгын хязгаарлалт нь Монголын жижиг, дунд бизнесүүдэд 
    саад болж болно. Интеграцын боломж өндөр боловч нээлттэй эхийн шинж чанаргүй.
    \item \textbf{BambooHR}: Жижиг, дунд бизнесүүдэд зориулсан энгийн, хэрэглэгчдэд ээлтэй гүйцэтгэлийн үнэлгээний модультай. 
    Гэсэн хэдий ч 360 хэмжээний санал хүсэлт, бодит цагийн аналитикийн хувьд хязгаарлагдмал бөгөөд интеграцын боломж сул. 
    Монголын зах зээлд нэвтрүүлэхэд тохируулга шаардлагатай.
    \item \textbf{EPES}: Бодит цагийн мэдээлэлд суурилсан үнэлгээ, нээлттэй эхийн технологи, өндөр тохируулгатай байдал зэргээрээ онцлог. 
    Docker болон Kubernetes-ийн дэмжлэг нь том хэмжээний байгууллагуудад тохиромжтой болгодог. Гэсэн хэдий ч хөгжүүлэлтийн эхний шатандаа 
    байгаа тул туршилт, баталгаажуулалт шаардлагатай. REST API болон нээлттэй эхийн шинж чанар нь ирээдүйд гуравдагч талын интеграцыг 
    өргөжүүлэх боломжтой.
\end{itemize}

\subsection{EPES-ийн боломжит сайжруулалт}
EPES системийн одоогийн хөгжүүлэлтийн байдлыг харгалзан үзэхэд дараах чиглэлээр сайжруулалт хийх боломжтой:
\begin{itemize}
    \item \textbf{Аналитикийн модуль}: Нарийвчилсан KPI болон OKR-д суурилсан аналитикийн алгоритмуудыг нэмж, гүйцэтгэлийн урьдчилсан таамаглал хийх боломжтой болгох.
    \item \textbf{Интеграцын өргөтгөл}: Slack, Microsoft Teams зэрэг алдартай SaaS платформтой шууд интеграцлах боломжийг нэмэх.
    \item \textbf{Хэрэглэгчийн туршлага}: Lattice-ийн түвшний UI/UX-ийг хангахын тулд хэрэглэгчийн санал хүсэлтэд суурилсан туршилт хийх.
    \item \textbf{Хамгаалалт}: JWT-ийн хажуугаанд OAuth 2.0 болон нэмэлт шифрлэлтийн протоколуудыг нэвтрүүлж, өгөгдлийн аюулгүй байдлыг сайжруулах.
\end{itemize}

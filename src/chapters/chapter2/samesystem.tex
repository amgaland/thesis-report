\section{Ижил төстэй системүүдийг судлах}
Эндээс бид дипломын ажлын хүрээнд хөгжүүлж буй системтэй ижил төстэй үйл ажиллагаа явуулдаг хоёр системийг судалж, тэдгээрийг харьцуулна. Судалгаанд Interactive.mn-ийн бүтээгдэхүүн болон Asana-ийн онцлогуудыг авч үзэх бөгөөд эдгээр системүүдийн байгууллагын танилцуулга, ажилтны гүйцэтгэлийг үнэлэх үндсэн модулиудыг тодорхойлно. Дараа нь эдгээр системүүдийг хөгжүүлж буй системтэй харьцуулж, давуу тал, онцлогийг шинжилнэ.

\subsection{Interactive.mn-ийн Бүтээгдэхүүн}
\begin{itemize}
    \item \textbf{Байгууллагын товч танилцуулга}: Interactive.mn нь Монгол Улсад байрладаг технологийн компани бөгөөд бизнесийн байгууллагуудад зориулсан программ хангамжийн шийдэл санал болгодог. Тус компанийн гол бүтээгдэхүүн нь байгууллагын дотоод үйл ажиллагааг автоматжуулах, ажилтнуудын гүйцэтгэлийг хянахад чиглэсэн системүүдийг хамардаг. Вэбсайтаас (https://interactive.mn/product) харахад тэдний шийдэл нь хэрэглэгчдэд ээлтэй интерфэйс, өөрчлөн тохируулах боломжоороо онцлогтой.
    \item \textbf{Ажилтны гүйцэтгэлийг үнэлэх үндсэн модуль}: Interactive.mn-ийн систем нь ажилтнуудын гүйцэтгэлийг хянахад чиглэсэн модультай бөгөөд энэ нь ажлын төлөвлөгөө, гүйцэтгэлийн үзүүлэлтүүдийг (KPI) тодорхойлох, тогтмол тайлагнах боломжийг олгодог. Модуль нь менежерүүдэд ажилтны ажлын ахиц, үр дүнг хянах боломж олгодог бөгөөд энгийн тайлангуудыг үүсгэх боломжтой. Гэсэн хэдий ч тодорхой мэдээлэл хязгаарлагдмал тул уг модуль нь голчлон энгийн KPI-д суурилсан үнэлгээнд чиглэдэг гэж дүгнэж болно.
\end{itemize}

\subsection{Asana-ийн Онцлогууд}
\begin{itemize}
    \item \textbf{Байгууллагын товч танилцуулга}: Asana нь АНУ-ын Сан Франциско хотод байрладаг программ хангамжийн компани бөгөөд 2008 онд Дустин Московиц болон Жастин Розенштейн нар үүсгэн байгуулжээ (https://asana.com/features). Тус компани нь багуудын ажлыг зохион байгуулах, хянах, удирдахад зориулсан SaaS платформыг санал болгодог. 2025 оны байдлаар Asana нь дэлхий даяар 131,000 гаруй хэрэглэгчтэй бөгөөд томоохон байгууллагууд (Uber, IBM гэх мэт)-д өргөн хэрэглэгддэг.
    \item \textbf{Ажилтны гүйцэтгэлийг үнэлэх үндсэн модуль}: Asana нь шууд гүйцэтгэлийн үнэлгээний модульгүй боловч "Goals" (Зорилго) онцлог болон таскийн хяналтын системээр дамжуулан ажилтны гүйцэтгэлийг хянах боломжтой. Хэрэглэгчид тодорхой ажлуудыг хуваарилж, хугацаа тогтоож, ахицыг хянах боломжтой бөгөөд энэ нь ажилтны гүйцэтгэлийг шууд бусаар үнэлэхэд ашиглагддаг. Нэмж дурдахад, Asana-ийн тайлагнах хэрэгслүүд нь баг болон хувь хүний гүйцэтгэлийн статистикийг гаргахад тусалдаг.
\end{itemize}

\subsection{Харьцуулалт}
Эндээс бид Interactive.mn болон Asana-ийн системүүдийг хөгжүүлж буй системтэй харьцуулна. Энэхүү систем нь Golang (Gin, GORM, JWT), Next.js (Tailwind CSS), PostgreSQL, Docker зэрэг технологийг ашиглан хөгжүүлэгдэж байгаа бөгөөд ажилтны гүйцэтгэлийг бодит цагийн мэдээлэлд суурилан үнэлэхэд чиглэсэн модультай.

\subsubsection{Дэлгэрэнгүй харьцуулалтын хүснэгт}
\begin{table}[H]
\centering
\small
\begin{tabular}{|p{2.5cm}|p{3.5cm}|p{3.5cm}|p{3.5cm}|}
\hline
\textbf{Онцлог} & \textbf{Interactive.mn} & \textbf{Asana} & \textbf{EPES} \\
\hline
\textbf{Танилцуулга} & Монголын технологийн компани, бизнесийн шийдэл & АНУ-ын SaaS платформ, баг хамтын ажиллагаа & Дипломын ажлын хүрээнд хөгжүүлэгдсэн вэб систем \\
\hline
\textbf{Үнэлгээний модуль} & KPI-д суурилсан энгийн үнэлгээ & Таскийн хяналт, зорилго тогтоох & Бодит цагийн мэдээлэлд суурилсан үнэлгээ \\
\hline
\textbf{Технологи} & Тодорхой бус (вэбд суурилсан) & JavaScript, React гэх мэт & Golang, Next.js, PostgreSQL \\
\hline
\textbf{Интерфэйс} & Хэрэглэгчдэд ээлтэй & Маш сайн UI/UX & Tailwind CSS-ээр хариу үйлдэлтэй \\
\hline
\textbf{Бодит цагийн хяналт} & Хязгаарлагдмал & Хэсэгчлэн дэмждэг & Бүрэн дэмждэг \\
\hline
\textbf{Тохируулга} & Дунд зэрэг & Өндөр & Өндөр (нээлттэй эх) \\
\hline
\textbf{Аюулгүй байдал} & Тодорхой бус & JWT болон HTTPS & JWT ашигласан \\
\hline
\textbf{Хэрэглээний хүрээ} & Жижиг, дунд бизнес & Том байгууллага & Төрөл бүрийн байгууллага \\
\hline
\end{tabular}
\caption{Ижил төстэй системүүдийн дэлгэрэнгүй харьцуулалт}
\label{tab:detailed_comparison}
\end{table}

\subsubsection{Системүүдийн харьцуулалтын хүснэгт}
Системүүдийн онцлогуудыг илүү тодорхой харуулахын тулд "Тийм" эсвэл "Үгүй" хариултад суурилсан харьцуулалтыг доорх хүснэгтэд үзүүлэв:

\begin{table}[H]
\centering
\small
\begin{tabular}{|>{\raggedright\arraybackslash}p{4cm}|c|c|c|}
\hline
\textbf{Онцлог} & \textbf{Interactive.mn} & \textbf{Asana} & \textbf{EPES} \\
\hline
Ажилтны гүйцэтгэлийг шууд үнэлэх модультай юу? & Тийм & Үгүй & Тийм \\
\hline
Бодит цагийн мэдээлэлд суурилсан уу? & Үгүй & Үгүй & Тийм \\
\hline
Нээлттэй эхийн технологи ашигласан уу? & Үгүй & Үгүй & Тийм \\
\hline
KPI-д суурилсан үнэлгээ хийдэг үү? & Тийм & Хэсэгчлэн & Тийм \\
\hline
Таскийн хяналтын системтэй юу? & Тийм & Тийм & Тийм \\
\hline
Хэрэглэгчийн интерфэйс хариу үйлдэлтэй юу? & Тийм & Тийм & Тийм \\
\hline
JWT аюулгүй байдлын стандарт ашигладаг уу? & Тодорхой бус & Тийм & Тийм \\
\hline
Docker-ээр контейнержсэн уу? & Үгүй & �Üгүй & Тийм \\
\hline
\end{tabular}
\caption{Ижил төстэй системүүдийн "Тийм/Үгүй" харьцуулалт}
\label{tab:yes_no_comparison}
\end{table}

\textbf{Шинжилгээ}:
\begin{itemize}
    \item \textbf{Interactive.mn}: Энгийн KPI-д суурилсан үнэлгээний модультай боловч бодит цагийн хяналт, нээлттэй эхийн технологи дутагдалтай. Жижиг, дунд бизнесүүдэд тохиромжтой ч том хэмжээний өргөтгөлд хязгаарлалттай.
    \item \textbf{Asana}: Таскийн хяналт, зорилго тогтоох онцлог нь гүйцэтгэлийг шууд бусаар үнэлэх боломжтой ч шууд модульгүй. Том байгууллагуудад тохиромжтой, гэхдээ бодит цагийн шинэчлэлт хязгаарлагдмал.
    \item \textbf{EPES}: Бодит цагийн мэдээлэлд суурилсан үнэлгээний модуль, нээлттэй эхийн технологи, Docker-ийн дэмжлэгээрээ онцлог. Энэ нь илүү уян хатан, тохируулгатай боловч хөгжүүлэлтийн эхний шатандаа тул том хэмжээний туршилт шаардлагатай.
\end{itemize}
\newpage
\section{Ижил төстэй системүүд}
Дипломын ажлын хүрээнд хөгжүүлж буй системтэй ижил төстэй үйл ажиллагаа явуулдаг 
хоёр системийг нарийвчлан судалж, тэдгээрийг харьцуулна. Судалгаанд Монголын 
Interactive.mn-ийн бүтээгдэхүүн болон олон улсын Asana-ийн онцлогуудыг авч үзэх 
бөгөөд эдгээр системүүдийн байгууллагын танилцуулга, ажилтны гүйцэтгэлийг үнэлэх
үндсэн модулиудыг тодорхойлно. Дараа нь эдгээр системүүдийг хөгжүүлж буй EPES 
(Employee Performance Evaluation System) системтэй харьцуулж, давуу тал, онцлог, 
болон боломжит хязгаарлалтуудыг шинжилнэ. Энэхүү харьцуулалт нь системийн функциональ
 болон техникийн шинж чанаруудыг тодруулахад чиглэнэ.

\subsection{Interactive.mn-ийн Бүтээгдэхүүн}
\begin{itemize}
    \item \textbf{Байгууллагын товч танилцуулга}: Interactive.mn нь Монгол Улсад байрладаг технологийн 
    компани бөгөөд бизнесийн байгууллагуудад зориулсан программ хангамжийн шийдэл санал болгодог. Тус 
    компанийн гол бүтээгдэхүүн нь байгууллагын дотоод үйл ажиллагааг автоматжуулах, ажилтнуудын 
    гүйцэтгэлийг хянах, мөн өдөр тутмын үйл ажиллагааг хялбаршуулахад чиглэсэн системүүдийг хамардаг. 
    Вэбсайтаас (\url{https://interactive.mn/product}) харахад тэдний шийдэл нь хэрэглэгчдэд ээлтэй 
    интерфэйс, өөрчлөн тохируулах боломж, болон хялбар нэвтрүүлэлтийн онцлогтой. Компани нь жижиг 
    болон дунд хэмжээний бизнесүүдэд голчлон үйлчилдэг бөгөөд Монголын зах зээлд тохирсон шийдэл 
    санал болгодог.
    \item \textbf{Ажилтны гүйцэтгэлийг үнэлэх үндсэн модуль}: Interactive.mn-ийн систем нь ажилтнуудын 
    гүйцэтгэлийг хянахад чиглэсэн модультай бөгөөд энэ нь ажлын төлөвлөгөө боловсруулах, гүйцэтгэлийн 
    үзүүлэлтүүдийг (KPI) тодорхойлох, мөн тогтмол тайлагнах боломжийг олгодог. Модуль нь менежерүүдэд 
    ажилтны ажлын ахиц, үр дүнг хянах боломж олгодог бөгөөд энгийн тайлангуудыг үүсгэх боломжтой. 
    Гэсэн хэдий ч уг модуль нь голчлон энгийн KPI-д суурилсан үнэлгээнд чиглэдэг бөгөөд нарийвчилсан 
    аналитик, бодит цагийн шинэчлэлт, эсвэл олон талт интеграцын боломжууд хязгаарлагдмал байж болно. 
    Энэ нь жижиг хэмжээний байгууллагуудад тохиромжтой боловч том байгууллагуудын нарийн шаардлагыг 
    хангахад хангалтгүй байж болох юм.
\end{itemize}

\subsection{Asana-ийн Онцлогууд}
\begin{itemize}
    \item \textbf{Байгууллагын товч танилцуулга}: Asana нь АНУ-ын Сан Франциско хотод байрладаг 
    программ хангамжийн компани бөгөөд 2008 онд Дустин Московиц болон Жастин Розенштейн нар үүсгэн 
    байгуулжээ (\url{https://asana.com/features}). Тус компани нь багуудын ажлыг зохион байгуулах, 
    хянах, удирдахад зориулсан SaaS (Software as a Service) платформыг санал болгодог. 
    2025 оны байдлаар Asana нь дэлхий даяар 131,000 гаруй байгууллага, 2.7 сая гаруй хэрэглэгчтэй 
    бөгөөд томоохон компаниуд (Uber, IBM, NASA гэх мэт)-д өргөн хэрэглэгддэг. Asana-ийн гол онцлог 
    нь хэрэглэгчдэд ээлтэй интерфэйс, өндөр тохируулгатай ажлын урсгал, болон бусад гуравдагч 
    талын хэрэгсэл (Slack, Google Drive гэх мэт)-тэй интеграцлах чадвар юм.
    \item \textbf{Ажилтны гүйцэтгэлийг үнэлэх үндсэн модуль}: Asana нь шууд гүйцэтгэлийн 
    үнэлгээний модульгүй боловч "Goals" (Зорилго) онцлог болон таскийн хяналтын системээр 
    дамжуулан ажилтны гүйцэтгэлийг хянах боломжтой. Хэрэглэгчид тодорхой ажлуудыг хуваарилж, 
    хугацаа тогтоож, ахицыг хянах боломжтой бөгөөд энэ нь ажилтны гүйцэтгэлийг шууд бусаар 
    үнэлэхэд ашиглагддаг. Нэмж дурдахад, Asana-ийн тайлагнах хэрэгслүүд нь баг болон хувь 
    хүний гүйцэтгэлийн статистикийг гаргахад тусалдаг бөгөөд "Workload" онцлог нь ажилтнуудын 
    ажлын ачааллыг хянах боломжийг олгодог. Гэсэн хэдий ч, Asana-ийн систем нь гүйцэтгэлийн 
    үнэлгээний тусгай модульгүй тул нарийвчилсан KPI-д суурилсан үнэлгээ эсвэл бодит цагийн 
    аналитикт чиглэсэн биш юм.
\end{itemize}

\subsection{Харьцуулалт}
Эндээс бид Interactive.mn болон Asana-ийн системүүдийг дипломын ажлын хүрээнд хөгжүүлж буй 
EPES системтэй харьцуулна. EPES нь Golang (Gin, GORM, JWT), Next.js (Tailwind CSS), 
PostgreSQL, Docker зэрэг орчин үеийн технологиудыг ашиглан хөгжүүлэгдэж байгаа бөгөөд 
ажилтны гүйцэтгэлийг бодит цагийн мэдээлэлд суурилан үнэлэхэд чиглэсэн модультай. 
Энэхүү харьцуулалт нь функциональ болон техникийн онцлогуудыг хамарна.

\subsubsection{Дэлгэрэнгүй харьцуулалтын хүснэгт}
\begin{table}[h]
\centering
\small
\begin{tabular}{|>{\raggedright\arraybackslash}p{3cm}|>{\raggedright\arraybackslash}p{4cm}|>{\raggedright\arraybackslash}p{4cm}|>{\raggedright\arraybackslash}p{4cm}|}
\hline
\textbf{Онцлог} & \textbf{Interactive.mn} & \textbf{Asana} & \textbf{EPES} \\
\hline
\textbf{Танилцуулга} & Монголын технологийн компани, бизнесийн шийдэл & АНУ-ын SaaS платформ, баг хамтын ажиллагаа & Дипломын ажлын хүрээнд хөгжүүлэгдсэн вэб систем \\
\hline
\textbf{Үнэлгээний модуль} & KPI-д суурилсан энгийн үнэлгээ & Таскийн хяналт, зорилго тогтоох & Бодит цагийн мэдээлэлд суурилсан үнэлгээ \\
\hline
\textbf{Технологи} & Тодорхой бус (вэбд суурилсан) & JavaScript, React, Python & Golang, Next.js, PostgreSQL \\
\hline
\textbf{Интерфэйс} & Хэрэглэгчдэд ээлтэй & Маш сайн UI/UX & Tailwind CSS-ээр хариу үйлдэлтэй \\
\hline
\textbf{Бодит цагийн хяналт} & Хязгаарлагдмал & Хэсэгчлэн дэмждэг & Бүрэн дэмждэг \\
\hline
\textbf{Тохируулга} & Дунд зэрэг & Өндөр & Өндөр (нээлттэй эх) \\
\hline
\textbf{Аюулгүй байдал} & Тодорхой бус & JWT, HTTPS, OAuth & JWT, HTTPS \\
\hline
\textbf{Хэрэглээний хүрээ} & Жижиг, дунд бизнес & Том байгууллага & Төрөл бүрийн байгууллага \\
\hline
\textbf{Интеграцын боломж} & Хязгаарлагдмал & Slack, Google Drive гэх мэт & REST API-аар дэмжигдэнэ \\
\hline
\textbf{Скалируемость} & Дунд зэрэг & Өндөр & Docker, Kubernetes-ээр өндөр \\
\hline
\end{tabular}
\caption{Ижил төстэй системүүдийн дэлгэрэнгүй харьцуулалт}
\label{tab:detailed_comparison}
\end{table}


\subsubsection{Интеграцын Харьцуулалт}
Системүүдийн өргөтгөх боломж болон интеграцын чадварыг илүү гүнзгий харьцуулахын тулд доорх хүснэгтийг нэмэв:

\begin{table}[h]
\centering
\small
\begin{tabular}{|>{\raggedright\arraybackslash}p{3.5cm}|>{\raggedright\arraybackslash}p{4cm}|>{\raggedright\arraybackslash}p{4cm}|>{\raggedright\arraybackslash}p{4cm}|}
\hline
\textbf{Онцлог} & \textbf{Interactive.mn} & \textbf{Asana} & \textbf{EPES} \\
\hline
\textbf{Хэрэглэгчийн хэмжээ} & Жижиг, дунд бизнес & Том байгууллага & Төрөл бүрийн хэмжээтэй \\
\hline
\textbf{Серверын архитектур} & Тодорхой бус & Cloud-based, өндөр скалируемость & Microservices, Docker, Kubernetes \\
\hline
\textbf{Интеграцын API} & Хязгаарлагдмал & REST API, Zapier & REST API \\
\hline
\textbf{Гуравдагч талын хэрэгсэл} & Хязгаарлагдмал & Slack, Google Drive, Microsoft Teams & Потенциалтай (нээлттэй эх) \\
\hline
\textbf{Өгөгдлийн хэмжээний дэмжлэг} & Дунд зэрэг & Өндөр & Өндөр (PostgreSQL) \\
\hline
\end{tabular}
\caption{Интеграцын харьцуулалт}
\label{tab:scalability_integration}
\end{table}

\subsection{Шинжилгээ ба Дүгнэлт}
\begin{itemize}
    \item \textbf{Interactive.mn}: Энгийн KPI-д суурилсан үнэлгээний модуль нь жижиг, дунд бизнесүүдэд тохиромжтой боловч бодит цагийн хяналт, нээлттэй эхийн технологи, болон өргөн хүрээтэй интеграцын боломжууд дутагдалтай. Системийн техникийн дэлгэрэнгүй мэдээлэл хязгаарлагдмал тул том хэмжээний өргөтгөлд хязгаарлалттай байж болно.
    \item \textbf{Asana}: Таскийн хяналт, зорилго тогтоох онцлог нь гүйцэтгэлийг шууд бусаар үнэлэх боломжтой боловч тусгай гүйцэтгэлийн үнэлгээний модульгүй. Том байгууллагуудад тохиромжтой, гэхдээ бодит цагийн шинэчлэлт болон нарийвчилсан KPI-д суурилсан аналитик хязгаарлагдмал. Asana-ийн интеграцын боломжууд болон хэрэглэгчийн интерфэйс нь өндөр чанартай.
    \item \textbf{EPES}: Бодит цагийн мэдээлэлд суурилсан үнэлгээний модуль, нээлттэй эхийн технологи, Docker болон Kubernetes-ийн дэмжлэгээрээ онцлог. Энэ нь илүү уян хатан, тохируулгатай боловч хөгжүүлэлтийн эхний шатандаа тул том хэмжээний туршилт, баталгаажуулалт шаардлагатай. REST API болон нээлттэй эхийн шинж чанар нь гуравдагч талын хэрэгсэлтэй интеграцлах боломжийг нэмэгдүүлнэ.
\end{itemize}

\subsection{EPES-ийн Боломжит Сайжруулалт}
EPES системийн одоогийн хөгжүүлэлтийн байдлыг харгалзан үзэхэд дараах чиглэлээр сайжруулалт хийх боломжтой:
\begin{itemize}
    \item \textbf{Аналитикийн модуль}: Нарийвчилсан KPI-д суурилсан аналитикын алгоритмуудыг нэвтрүүлж, гүйцэтгэлийн үнэлгээ хийх боломжтой болгох.
    \item \textbf{Интеграцын өргөтгөл}: Slack, Microsoft Teams зэрэг алдартай SaaS платформтой шууд интеграцлах боломжийг нэмэх.
    \item \textbf{Хэрэглэгчийн туршлага}: UI/UX-ийг Asana-ийн түвшинд хүргэхийн тулд хэрэглэгчийн санал хүсэлтэд суурилсан туршилт, сайжруулалт хийх.
    \item \textbf{Хамгаалалт}: JWT-ийн хажуугаанд OAuth 2.0 болон нэмэлт шифрлэлтийн протоколуудыг нэвтрүүлж, өгөгдлийн аюулгүй байдлыг сайжруулах.
\end{itemize}
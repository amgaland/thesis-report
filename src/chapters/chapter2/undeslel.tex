\section{Үндэслэл}

Бизнесийн байгууллагын ажилчны гүйцэтгэлийг үнэлэх системийн сэдвийг сонгох нь нийгмийн, эдийн засгийн,
технологийн шалтгаан нөхцөл, шаардлага, түүнчлэн энэхүү судалгааны үр нөлөөтэй нягт холбоотой юм. 
Эдгээр хүчин зүйлсийг доор тодорхой тайлбарлав.

\subsection{Нийгмийн шалтгаан нөхцөл}

Орчин үеийн нийгэмд ажилтны гүйцэтгэлийн үнэлгээ нь зөвхөн байгууллагын зорилгод хүрэхээс 
гадна ажилтны хувь хүний хөгжил, сэтгэл ханамжийг дэмжихэд чухал үүрэг гүйцэтгэж байна. 
Монголын хөдөлмөрийн зах зээлд ажилтны ажлын байрны тогтвортой байдал, сэдэлжилд, ур чадварыг 
дээшлүүлэх нь нийгмийн хөгжлийн нэгэн чухал хэсэг болоод байна. Судалгаагаар, гүйцэтгэлийн 
үнэлгээний ил тод, шударга систем ашигладаг байгууллагуудын ажилтны ажлын сэтгэл ханамж 18 
хувиар өсдөг [1]. Иймээс энэхүү сэдвийг сонгосон нь нийгэмд ажилтны хөгжил, байгууллагын 
соёлыг сайжруулахад хувь нэмэр оруулах зорилготой юм.

\subsection{Эдийн засгийн шалтгаан нөхцөл}

Ажилчны гүйцэтгэлийн үнэлгээний систем нь байгууллагын эдийн засгийн үр ашгийг нэмэгдүүлэхэд шууд нөлөөтэй. 
Бизнесийн байгууллагуудын хувьд уламжлалт үнэлгээний аргууд нь цаг хугацаа их шаардлагатай, алдаа 
гарах эрсдэлтэй байдаг бөгөөд энэ нь ажилтан ажилдаа сэтгэл ханамжгүй байх мөн түүнчлэн ажил хаялтад хүргэдэг. 
Автоматжуулсан гүйцэтгэлийн үнэлгээний систем нэвтрүүлсэн байгууллагуудын үйл ажиллагааны зардал дунджаар 15 хувиар буурч, ажилтны бүтээмж 20 хувиар 
өсдөг болохыг олон улсын судалгаа харуулж байна [2]. Энэ сэдвийг сонгосон нь бизнесийн байгууллагуудын 
өрсөлдөх чадварыг нэмэгдүүлж, эдийн засгийн хувьд илүү үр ашигтай шийдэл санал болгоход чиглэгдэнэ.

\subsection{Технологийн шалтгаан нөхцөл}

Мэдээллийн технологийн хурдацтай хөгжил нь бизнесийн процессуудыг автоматжуулах, өгөгдөлд суурилсан шийдвэр 
гаргалтыг нэмэгдүүлэх боломжийг олгож байна. Next.js, Golang, PostgreSQL, Docker зэрэг орчин үеийн технологиудыг 
ашигласан вэбд суурилсан гүйцэтгэлийн үнэлгээний систем нь бодит цагийн мэдээлэл боловсруулалт, хэрэглэгчийн 
ээлтэй интерейсээр хангадаг. Монголын бизнесийн байгууллагуудын дийлэнх нь ийм технологийн 
шийдлийг хараахан нэвтрүүлээгүй байгаа нь энэ сэдвийн технологийн хэрэгцээг онцолж байна. Судалгаагаар, 
гүйцэтгэлийн үнэлгээний процессыг автоматжуулсанаар процессын хугацаа 35 хувиар буурдаг [3].



\subsection{Шаардлага}

Монголын бизнесийн орчинд гүйцэтгэлийн үнэлгээний автоматжуулсан системийн хэрэгцээ улам бүр нэмэгдэж байна. Уламжлалт аргууд нь 
субьектив байдал, мэдээллийн алдагдал, процессын удаан байдал зэрэг асуудлуудыг дагуулдаг. Түүнчлэн 360 хэмжээний санал хүсэлт, OKR, 
KPI зэрэг орчин үеийн үнэлгээний аргуудыг нэвтрүүлэх шаардлага Монголын байгууллагуудад тулгарч байна. Энэхүү судалгаа нь эдгээр шаардлагыг 
хангах, Монголын бизнесийн онцлогт тохирсон, найдвартай, хялбар хэрэглэгдэхүйц системийн загварыг боловсруулахыг зорьж байна.

\subsection{Үр нөлөө}

Энэхүү судалгааны үр дүнд боловсруулагдсан гүйцэтгэлийн үнэлгээний систем нь дараах үр нөлөөг бий болгоно:

\begin{itemize}
    \item \textbf{Нийгмийн хувьд}: ажилтны ажлын сэтгэл ханамж, сэдэлжилтийг дээшлүүлж, ажилтны хөгжил, байгууллагын соёлыг сайжруулна.
    \item \textbf{Эдийн засгийн хувьд}: Байгууллагын үйл ажиллагааны зардлыг бууруулж, бүтээмжийг нэмэгдүүлснээр эдийн засгийн үр ашгийг дээшлүүлнэ.
    \item \textbf{Технологийн хувьд}: Монголын мэдээллийн технологийн салбарт шинэ загвар, технологийн шийдлийг нэвтрүүлж, дижитал шилжилтийг дэмжинэ.
    \item \textbf{Практикийн хувьд}: Байгууллагуудын гүйцэтгэлийн үнэлгээний процессыг ил тод, хурдан, үр ашигтай болгоно.
\end{itemize}

Эдгээр шалтгаан, шаардлага, үр нөлөөний үндсэн дээр "Бизнесийн байгууллагын ажилчны гүйцэтгэлийг үнэлэх систем" сэдвийг сонгосон нь Монголын 
бизнесийн орчинд шинжлэх ухаан, практикийн хувьд онцгой ач холбогдолтой юм.

\section{Онолын судалгаа}

Орчин үеийн байгууллагууд ажилтны ур чадвар, ажлын чанар, байгууллагын зорилгод оруулж буй 
хувь нэмрийг үнэлэх зорилгоор гүйцэтгэлийн үнэлгээний системийг тогтмол хэрэгжүүлж байна. 
Гүйцэтгэлийн үнэлгээ нь зөвхөн шагнал урамшуулал, тушаал дэвшүүлэлт, цалин нэмэгдүүлэлт зэрэг 
хүний нөөцийн шийдвэр гаргахад хэрэглэгдээд зогсохгүй, байгууллагын урт хугацааны стратегийн 
төлөвлөлтөд чухал үүрэгтэй. Энэхүү судалгаанд уламжлалт болон орчин үеийн гүйцэтгэлийн үнэлгээний 
арга, хэрэгсэл, давуу болон сул талыг харьцуулан судалж, AHP\footnote{AHP - Analytical Hierarchy Process} 
зэрэг орчин үеийн шийдвэр гаргалтын аргачлалын давуу талыг ашиглан тодотгож өгнө.


\begin{enumerate}
    \item \textbf{Гүйцэтгэлийн үнэлгээний ерөнхий зорилго}
    \begin{itemize}
        \item Байгууллагын зорилгод хүрэхэд ажилтны хувь нэмрийг үнэлэх
        \item Шагнал урамшуулал, карьерын өсөлт, цалинтай амралт зэрэг хүний нөөцийн шийдвэрт туслах
        \item Гүйцэтгэлийг нэмэх зорилготой хөгжлийн төлөвлөгөөг боловсруулах
        \item Ур чадварын зөрүүг илрүүлэх, түүнийг нөхөхөд хийгдэх сургалт, хөгжлийн төлөвлөгөө боловсруулах
    \end{itemize}
    \item \textbf{Уламжлалт аргачилалууд}
    \begin{enumerate}
        \item \textbf{Эрэмбэлэх арга (Ranking method):}  Ажилтнуудыг шууд даргын зүгээс хамгийн сайн нь хэн болохыг 
        харьцуулан эрэмбэлдэг. Гэвч үнэлгээний үндэслэл тодорхой бус, субъектив шинжтэй.
        \begin{itemize}
            \item Давуу тал
            \begin{itemize}
                \item Энгийн бөгөөд ашиглахад хялбар.
                \item Хурдан бөгөөд ил тод.
            \end{itemize}
            \item Сул тал
            \begin{itemize}
                \item Объектив байдал нь бага 
                \item Олон ажилтантай байгууллагад тохиромжгүй
                \item Ажилтны давуу болон сул талыг тодорхойлоход хүндрэлтэй
            \end{itemize}
        \end{itemize}
        \item \textbf{График үнэлгээний хуваарь (Graphic Rating Scales):}Ажилтныг хэд хэдэн чанарын дагуу 
        (жишээлбэл, харилцаа, ажлын гүйцэтгэл) тодорхой үнэлгээний шалгуураар дүгнэдэг.
        \begin{itemize}
            \item Давуу тал
            \begin{itemize}
                \item Дасан зохицох чадвартай
                \item Ашиглахад болон боловсруулахад хялбар
                \item Зардал багатай
                \item Бүх төрлийн ажлыг үнэлж болно
                \item Олон тооны ажилтныг хамарч чадна
            \end{itemize}
            \item Сул тал
            \begin{itemize}
                \item Үнэлэгээ гаргагчийн хувийн хандлага (субъектив байдал)
                \item Бүх шалгуурыг адил жинтэйд тооцдог
            \end{itemize}
        \end{itemize}
        \item \textbf{Чухал тохиолдлын арга (Critical Incident Method):}Тухайн ажилтны ажлын явцад гаргасан онцгой, 
        эерэг болон сөрөг зан төлөвийг тэмдэглэж, түүний дагуу үнэлгээ хийдэг.
        \begin{itemize}
            \item Давуу тал
            \begin{itemize}
                \item Санал хүсэлт өгөхөд хялбар
                \item Үнэлгээ нь бодит ажлын зан төлөвт үндэслэдэг
                \item Доод албан тушаалтнуудын сайжрах боломж өндөр
            \end{itemize}
            \item Сул тал
            \begin{itemize}
                \item Мэдээллийг шинжлэх, нэгтгэхэд их цаг зарцуулдаг
                \item Судалгаагаар чухал үйл явдлын мэдээлэл цуглуулах нь хүндрэлтэй
            \end{itemize}
        \end{itemize}
        \item \textbf{Narrative Essay:} Удирдах албан тушаалтан ажилтны давуу, сул талыг бичгээр тайлбарлаж, хөгжүүлэх чиглэл өгөх зорилгоор хэрэглэнэ.
        \begin{itemize}
            \item Давуу тал
            \begin{itemize}
                \item Ажилтантай холбоотой мэдээллийн хоосон зайг нөхдөг
                \item Бүх хүчин зүйлийг хамардаг
                \item Дэлгэрэнгүй, цогц санал хүсэлт өгдөг
            \end{itemize}
            \item Сул тал
            \begin{itemize}
                \item Цаг их шаарддаг
                \item Үнэлэгчийн хувийн хандлагад амархан автдаг
                \item Үр дүнтэй бичиж чаддаг үнэлэгээ гаргагч шаардлагатай
            \end{itemize}
        \end{itemize}

    \end{enumerate}
    \item \textbf{Орчин үеийн аргачилалууд}
    \begin{enumerate}
        \item \textbf{Зорилгоор удирдах арга (Management by Objectives – MBO):} Ажилтны гүйцэтгэлийг 
        удирдлагын зүгээс тодорхойлсон зорилтуудын хэрэгжилттэй харьцуулан үнэлдэг. Зорилт тогтоох, 
        хэрэгжүүлэх, санал хүсэлт өгөх гурван үндсэн үйл явцтай. Weihrich MBO-г системчилсэн 7 үе 
        шаттайгаар тайлбарласан.
        \begin{itemize}
            \item Давуу тал
            \begin{itemize}
                \item Хэрэгжүүлэх болон хэмжихэд хялбар
                \item Ажилтнуудын үүрэг, хариуцлагыг тодорхой ойлгуулах боломжтой
                \item Ажилтанд зөвлөгөө өгөх, чиглүүлэхэд дэмжлэг болдог
            \end{itemize}
            \item Сул тал
            \begin{itemize}
                \item Зорилгыг өөрөөр ойлгоход ойлголтын зөрүү гарч болзошгүй
                \item Шударга байдал, чанар зэрэг чухал үнэт зүйлсийг орхигдуулах эрсдэлтэй
                \item Үнэлүүлж буй ажилтан зорилгод санал нийлэхгүй байх магадлалтай
                \item Ажлын бүх төрлөд тохиромжтой биш
            \end{itemize}
        \end{itemize}
        \item \textbf{Зан үйлийн үнэлгээний шкал (Behaviorally Anchored Rating Scales – BARS):} Хувь хүний 
        гүйцэтгэлийг зан төлөвийн жишээн дээр үндэслэн тодорхойлж, тоон үнэлгээтэй уялдуулан дүгнэдэг.
        \begin{itemize}
            \item Давуу тал
            \begin{itemize}
                \item Ажилтны гүйцэтгэлийг мэргэжлийн үүднээс ажлын зан төлөвөөр тодорхойлдог
                \item Үнэлэгч ба үнэлүүлж буй хүн хамтран оролцсоноор үнэлгээг илүү хүлээн зөвшөөрөх магадлалтай
                \item Үнэлгээний алдааг багасгахад тусалдаг
            \end{itemize}
            \item Сул тал
            \begin{itemize}
                \item Хэмжээст хамааралгүй байдал нь зарим тохиолдолд хүчинтэй эсвэл найдвартай биш байж магадгүй
                \item Зан төлөв нь үр дүн гэхээсээ илүү үйл ажиллагаанд чиглэсэн байдаг
                \item Цаг их шаарддаг
                \item Ажлын төрөл бүр тусдаа BARS (Behaviorally Anchored Rating Scale) хэмжих шаардлагатай
            \end{itemize}
        \end{itemize}
        \item \textbf{Хүний нөөцийн бүртгэл, тооцоолол (Human Resource Accounting – HRA):} Ажилтны 
        байгууллагад оруулж буй бодит хувь нэмэр болон өртгийг үнэлж, нягтлан бодох бүртгэлийн 
        аргачлалаар илэрхийлдэг.
        \begin{itemize}
            \item Давуу тал
            \begin{itemize}
                \item Хүний нөөцийг сайжруулах боломж олгодог
                \item Хүний нөөцийн бодлогыг боловсруулах, хэрэгжүүлэхэд тусалдаг
                \item Хүний нөөцөд хийсэн хөрөнгө оруулалтын үр өгөөжийг үнэлдэг
                \item Ажилтны ур чадвар, чадамжийг дээшлүүлэхэд чиглэгддэг
            \end{itemize}
            \item Сул тал
            \begin{itemize}
                \item Хүний нөөцийн зардал ба үнэ цэнийг тодорхойлох тодорхой зааварчилгаа дутмаг
                \item Зөвхөн байгууллагын зардлыг хэмждэг бөгөөд ажилтны байгууллагад оруулж буй бодит үнэ цэнийг тооцдоггүй
                \item Тодорхойгүй нөхцөл байдалд ажилтны гүйцэтгэлийг бодитоор хэмжих нь бодит бус байдаг
            \end{itemize}
        \end{itemize}
        \item \textbf{Үнэлгээний төв (Assessment Center):} Ажилтныг мэргэжлийн ажиглагчдаар ажлын орчны
         дасгал, симуляци, бүлгийн хэлэлцүүлгээр дамжуулан үнэлдэг төвлөрсөн үнэлгээний хэлбэр.
        \begin{itemize}
            \item Давуу тал
            \begin{itemize}
                \item Ирээдүйн гүйцэтгэл, ахиц дэвшлийг илүү нарийн таамаглах боломжтой
                \item Үндсэн ойлголтууд нь энгийн
                \item Уян хатан аргачлалтай
                \item Албан тушаал дэвшүүлэх шийдвэр гаргалт болон ажилтны хөгжилд шаардлагатай хэрэгцээг тодорхойлоход дэмжлэг үзүүлдэг
                \item Олон төрлийн шинж чанарыг зэрэг үнэлэх боломжтой
            \end{itemize}
            \item Сул тал
            \begin{itemize}
                \item Зардал өндөртэй, удирдахад хүндрэлтэй
                \item Олон ажилтан, их хэмжээний цаг хугацаа шаарддаг
                \item Нэг дор цөөн тооны хүнийг л үнэлэх боломжтой
            \end{itemize}
        \end{itemize}
        \item \textbf{360 хэмийн үнэлгээ (360 Degree Feedback):} Дарга, багийн гишүүд, хэрэглэгчид, 
        хамт олон болон өөрийн үнэлгээ зэрэг олон талаас мэдээлэл авч, ажилтныг иж бүрнээр үнэлнэ.
        \begin{itemize}
            \item Давуу тал
            \begin{itemize}
                \item Ажилтнууд өдөр тутам харилцдаг хүмүүст үзүүлж буй нөлөөгөө илүү сайн ойлгох боломжтой
                \item Ажилтны хөгжлийн маш сайн хэрэгсэл болдог
                \item Нарийн, найдвартай систем
            \end{itemize}
            \item Сул тал
            \begin{itemize}
                \item Цаг хугацаа их шаарддаг, зардал өндөртэй
                \item Өөр өөр бүлгүүдийн дүн шинжилгээ зөрүүтэй гарвал тайлбарлахад хүндрэлтэй
                \item Хэлтсүүдийн хоорондын (cross-functional) багуудад хэрэгжүүлэхэд хүндрэлтэй
                \item Нууцлалыг хадгалах нь бэрхшээлтэй
            \end{itemize}
        \end{itemize}
    \end{enumerate}
\end{enumerate}

Ажилтны гүйцэтгэлийн үнэлгээ бол байгууллагын хүний нөөцийн бодлого, хөгжлийн гол хэрэгсэл юм.
 Үнэлгээ нь шударга, ил тод, системтэй байж чадвал байгууллагын бүтээмжид үнэтэй хувь нэмэр оруулна.
  Уламжлалт арга нь хялбар боловч субъектив, харин орчин үеийн арга нь илүү иж бүрэн, үнэн зөв, 
  оролцоонд суурилсан байдаг. AHP зэрэг олон шалгуурт шийдвэр гаргалтын арга нь гүйцэтгэлийг 
  илүү нарийвчлалтай, шударга үнэлэх боломжийг нээдэг.

Тиймээс байгууллагууд өөрсдийн онцлогт тохируулан гүйцэтгэлийн үнэлгээний системээ сайтар боловсруулж,
 тогтмол шинэчилж байх нь зүйтэй.
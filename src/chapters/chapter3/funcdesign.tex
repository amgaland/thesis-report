% functional zagvar end bichne

Функциональ загвар нь системийн гол бүрэлдэхүүн хэсгүүд болон тэдгээрийн харилцан үйлчлэлийг тодорхойлж, систем хэрхэн ажиллахыг графикаар харуулна. Энэ загвар нь хөгжүүлэлтийн үндсэн суурь болж, системийн архитектурыг ойлгоход тусална. Загварт дараах бүрэлдэхүүн хэсгүүд багтана:
\begin{itemize}
    \item \textbf{Хэрэглэгчийн Интерфэйс}: Next.js болон Tailwind CSS ашиглан хэрэглэгчийн харагдац, харилцан үйлдэл хийх хэсэг.
    \item \textbf{API Сервер}: Gin фреймворк ашиглан RESTful API endpoint-уудыг хангаж, клиент болон мэдээллийн сан хоорондын харилцааг зохицуулна.
    \item \textbf{Мэдээллийн Сан}: PostgreSQL дээр GORM ашиглан мэдээллийн хадгалалт, удирдлагыг гүйцэтгэнэ.
    \item \textbf{Контейнержуулалт}: Docker ашиглан сервер болон мэдээллийн сангийн орчныг стандартжуулж, тогтвортой байдлыг хангана.
\end{itemize}
Эдгээр бүрэлдэхүүн хэсгүүд хоорондоо HTTP хүсэлт (REST API) болон SQL асуулгаар харилцана. Доорх диаграммд энэ харилцан үйлчлэлийг графикаар харуулав.

\subsubsection{Функциональ Загварын Диаграмм}
\begin{figure}[h]
\centering

\caption{Системийн Функциональ Загварын Диаграмм}
\label{fig:functional_model}
\end{figure}

Эндээс хэрэглэгчээс эхэлсэн хүсэлт хэрэглэгчийн интерфэйсээр дамжин API сервер руу очиж, улмаар мэдээллийн сантай харилцан үйлчлэл үйлдэгдэнэ. Docker нь API сервер болон мэдээллийн сангийн орчныг стандартжуулж, тогтвортой байдлыг хангана. Хариу нь эсрэг чиглэлд буцаж, хэрэглэгчид харагдана.

% Ensure UTF-8 encoding (save as UTF-8 without BOM)
\documentclass[a4paper]{report} % For standalone testing; remove when included
\usepackage{fontspec}
\usepackage{xltxtra}
\usepackage[mongolian]{babel}
\usepackage{geometry}
\usepackage{array}
\usepackage{float}
\usepackage{longtable}
\geometry{a4paper,margin=1in}
\setmainfont{Times New Roman}

\begin{document}

\subsection{Функциональ Шаардлага}
Функциональ шаардлага нь системийн гол үйл ажиллагааг тодорхойлж, хэрэглэгчдэд ямар боломж олгохыг заана. Эдгээр шаардлага нь системийн зорилгыг хангаж, хэрэглэгчийн туршлагыг сайжруулан, гүйцэтгэлийн үнэлгээний процессыг автоматжуулна. Шаардлагуудыг ерөнхий болон хэрэглэгчийн төрлүүдээр (Администратор, Менежер, Ажилтан) ангилж, доорх хүснэгтэд дэлгэрэнгүй харуулав. Шаардлага бүрт код оноож, тодорхойлолтыг нарийвчлан өгснөөр хөгжүүлэлтийн явцад тодорхой бус байдлыг багасгана.

\subsubsection{Ерөнхий Шаардлага}
Ерөнхий шаардлагууд нь системийн суурь үйл ажиллагааг хамардаг бөгөөд бүх хэрэглэгчидтэй холбоотой үндсэн функцуудыг тодорхойлно. Эдгээр нь аюулгүй байдал, хэрэглэгчийн бүртгэл, мэдээллийн хадгалалтад чиглэнэ.

\begin{longtable}{|p{2cm}|p{8.5cm}|}
\hline
\textbf{Код} & \textbf{Функциональ Шаардлага} \\ \hline
\endhead
ФШ100 & Хэрэглэгч бүртгэх: Систем шинэ хэрэглэгчдийг бүртгэнэ (имэйл, нэр, дүр). Давхардлыг шалгана. \\ \hline
ФШ101 & Нэвтрэлтийн баталгаажуулалт: Хэрэглэгчид имэйл, нууц үгээр нэвтэрнэ. JWT ашиглана. \\ \hline
ФШ102 & Нууц үг сэргээх: Хэрэглэгч нууц үгээ мартвал имэйлээр сэргээх холбоос илгээнэ (хугацаатай токен). \\ \hline
ФШ103 & Системийн лог: Хэрэглэгчийн үйлдэл, алдааг PostgreSQL-д тэмдэглэнэ. \\ \hline
ФШ104 & Мэдээлэл хадгалах: Хэрэглэгчийн профайл (нэр, имэйл, дүр) PostgreSQL-д хадгалагдаж, засварлагдана. \\ \hline
ФШ105 & Сессийн удирдлага: JWT токеноор сессийг хянаж, 30 минут идэвхгүй бол дуусгана. \\ \hline
\caption{Ерөнхий Функциональ Шаардлагууд} \label{tab:general_requirements}
\end{longtable}

\subsubsection{Администраторын Шаардлага}
Администраторын шаардлагууд нь системийн удирдлага, аюулгүй байдал, засвар үйлчилгээтэй холбоотой бөгөөд тогтвортой байдлыг хангадаг.

\begin{longtable}{|p{2cm}|p{8.5cm}|}
\hline
\textbf{Код} & \textbf{Функциональ Шаардлага} \\ \hline
\endhead
ФШ200 & Эрхийн удирдлага: Администратор хэрэглэгчдэд дүр (Admin, Manager, Employee) оноож, хандалтыг тохируулна. \\ \hline
ФШ201 & Системийн тохиргоо: Параметрүүд (KPI-ийн жин, мэдэгдлийн давтамж) Next.js интерфэйсээр засварлагдана. \\ \hline
ФШ202 & Нөөцлөлт, сэргээлт: PostgreSQL-ийн мэдээллийг өдөр бүр нөөцөлж, Docker-д сэргээнэ. \\ \hline
ФШ203 & Үйлдлийн түүх: Нэвтрэлт, үнэлгээний оруулалтыг хугацаа, ID-тай хянаж, тайлан гаргана. \\ \hline
ФШ204 & Алдааны тайлан: API тасалдал, PostgreSQL холболтын алдааг тэмдэглэн, мэдэгдэнэ. \\ \hline
ФШ205 & Хэрэглэгч идэвхгүй болгох: Хэрэглэгчийг хасах эсвэл түр идэвхгүй болгоно. \\ \hline
\caption{Администраторын Функциональ Шаардлагууд} \label{tab:admin_requirements}
\end{longtable}

\subsubsection{Менежерын Шаардлага}
Менежерын шаардлагууд нь ажилтны гүйцэтгэлийг удирдах, хянах, тайлагнахад чиглэж, системийн гол зорилгыг хэрэгжүүлнэ.

\begin{longtable}{|p{2cm}|p{8.5cm}|}
\hline
\textbf{Код} & \textbf{Функциональ Шаардлага} \\ \hline
\endhead
ФШ300 & Даалгавар хуваарилах: Менежер даалгавар үүсгэж, хугацаа, тодорхойлолттой хуваарилна. \\ \hline
ФШ301 & Гүйцэтгэлийн хяналт: Даалгаврын статус (эхэлсэн, дууссан) бодит цагт хянагдана (жишээ: 75\%). \\ \hline
ФШ302 & KPI-д суурилсан үнэлгээ: Менежер KPI (хурд, чанар) оноо, санал оруулна. \\ \hline
ФШ303 & Тайлан гаргах: Гүйцэтгэлийн түүх, KPI оноог график хэлбэрээр харуулна. \\ \hline
ФШ304 & Тайлан экспорт: Тайланг PDF (график) эсвэл CSV (өгөгдөл) хэлбэрээр татна. \\ \hline
ФШ305 & Харьцуулалт: Багийн гишүүдийн гүйцэтгэлийг KPI оноогоор харьцуулж, шилдэгийг тодорхойлно. \\ \hline
\caption{Менежерын Функциональ Шаардлагууд} \label{tab:manager_requirements}
\end{longtable}

\subsubsection{Ажилтны Шаардлага}
Ажилтны шаардлагууд нь хувь хүний гүйцэтгэлийг хянах, даалгаврыг удирдахад чиглэж, идэвхийг дэмжинэ.

\begin{longtable}{|p{2cm}|p{8.5cm}|}
\hline
\textbf{Код} & \textbf{Функциональ Шаардлага} \\ \hline
\endhead
ФШ400 & Даалгавар харах: Ажилтан хуваарилагдсан даалгавруудын жагсаалтыг (хугацаа, тодорхойлолт) харна. \\ \hline
ФШ401 & Статус шинэчлэх: Даалгаврын явцыг (хийгдэж байна, дууссан) тэмдэглэнэ. \\ \hline
ФШ402 & Гүйцэтгэл хянах: Ажилтан KPI оноо, саналыг график хэлбэрээр харна. \\ \hline
ФШ403 & Мэдэгдэл: Даалгаврын хугацаа ойртох (24 цагийн өмнө) эсвэл хэтэрвэл мэдэгдэне. \\ \hline
ФШ404 & Профайл засах: Ажилтан мэдээлэл (имэйл, зураг), нууц үгээ өөрчилнө. \\ \hline
ФШ405 & Өөрийгөө үнэлэх: Ажилтан гүйцэтгэлээ үнэлж, санал оруулна (менежерт хянагдана). \\ \hline
\caption{Ажилтны Функциональ Шаардлагууд} \label{tab:employee_requirements}
\end{longtable}

\end{document}
% Ugugdliin sangiin zagvariig end bichne
% % database-model.tex
% \section*{Өгөгдлийн Сангийн Загвар}
% Эндээс бид системийн өгөгдлийн сангийн загварыг тодорхойлж, хүснэгтүүд болон тэдгээрийн харилцааг харуулна.

% \subsection*{Загварын Тойм}
% Өгөгдлийн сан нь PostgreSQL-д суурилсан бөгөөд "Хэрэглэгчид", "Даалгаврууд", "Гүйцэтгэл" гэсэн үндсэн хүснэгтүүдийг агуулна. Хүснэгтүүд хоорондоо гадаад түлхүүрээр холбогдож, хэрэглэгчийн мэдээлэл, даалгаврын статус, гүйцэтгэлийн үнэлгээг хадгална.

% \subsection*{ER Диаграм}
% \begin{figure}[ht]
%     \centering
%     \includegraphics[width=\textwidth]{src/images/er_diagram.png}
%     \caption{Ажилтны Гүйцэтгэлийн Үнэлгээний Системийн ER Диаграм}
%     \label{fig:er_diagram}
% \end{figure}

% \textbf{Тайлбар:} Диаграммд "Хэрэглэгчид" (id, нэр, дүр), "Даалгаврууд" (id, хэрэглэгчийн id, статус), "Гүйцэтгэл" (id, хэрэглэгчийн id, KPI оноо) хүснэгтүүдийн харилцааг харуулсан бөгөөд хэрэглэгч даалгавар хуваарилж, гүйцэтгэлийг ү
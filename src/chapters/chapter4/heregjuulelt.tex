\section{Хэрэгжүүлэлт}
Энэ хэсэгт ажилчны гүйцэтгэлийн үнэлгээний систем (EPES)-ийн хэрэгжүүлэлтийн үйл явцыг дэлгэрэнгүй тайлбарлаж, хөгжүүлэлтийн үе шатууд, ашигласан технологиудын хэрэглээ, 
тулгарсан асуудлууд болон тэдгээрийн шийдлийг тодорхойлно. Хэрэгжүүлэлтийн кодын хэсгүүдийг \hyperref[sec:appendix]{Хавсралт}-д тусгасан бөгөөд энэ хэсэгт кодын үйл ажиллагаа, 
хөгжүүлэлтийн үр дүнтэй хамаарлыг оновчтой харуулна.

\subsection{Хэрэгжүүлэлтийн үе шатууд}
EPES системийн хөгжүүлэлт дараах үндсэн үе шатуудаас бүрдсэн:
\begin{itemize}
    \item \textbf{Шаардлагын шинжилгээ}: Бизнесийн байгууллагуудын гүйцэтгэлийн үнэлгээний хэрэгцээг судалж, бодит цагийн хяналт, 360 хэмжээний санал хүсэлт, OKR болон 
    KPI-д суурилсан модулиудыг тодорхойлсон.
    \item \textbf{Технологийн сонголт}: Golang (Gin, GORM, JWT), Next.js (Tailwind CSS), PostgreSQL, Docker зэрэг технологиудыг сонгож, тэдгээрийн өндөр гүйцэтгэл, 
    хямд байдал, бизнесийн орчинд тохирсон байдлыг харгалзсан.
    \item \textbf{Системийн загварчлал}: REST API-д суурилсан микро үйлчилгээний архитектурыг бий болгож, front-end болон back-end харилцан ажиллагааг төлөвлөсөн.
    \item \textbf{Хөгжүүлэлт ба туршилт}: Системийн модулиудыг хөгжүүлж, Postman болон DBeaver ашиглан API болон өгөгдлийн сангийн гүйцэтгэлийг туршсан.
    \item \textbf{Нэвтрүүлэлт}: Docker-ийн тусламжтайгаар системийг тогтвортой орчинд нэвтрүүлсэн.
\end{itemize}

\subsection{Кодчиллын хэсгүүдийн хамаарал}
Хэрэгжүүлэлтийн кодчиллын гол хэсгүүдийг \hyperref[sec:appendix]{Хавсралт}-д оруулсан бөгөөд эдгээр нь системийн үндсэн модулиудын үйл ажиллагааг хангахад шууд хамааралтай юм:
\begin{itemize}
    \item \textbf{REST API (Golang, Gin)}: Хэрэглэгчийн баталгаажуулалт, гүйцэтгэлийн мэдээллийн асуулга, санал хүсэлтийн боловсруулалтыг хариуцдаг. 
    Жишээлбэл, \hyperref[lst:api]{Хавсралт А}-д оруулсан API endpoint нь KPI мэдээллийг бодит цагт авах боломжийг олгож, системийн хурд, найдвартай байдлыг хангана.
    \item \textbf{Өгөгдлийн сангийн удирдлага (GORM, PostgreSQL)}: Гүйцэтгэлийн өгөгдөл, санал хүсэлтийн түүхийг хадгалах, хурдан хайлт хийхэд зориулагдсан. 
    \hyperref[lst:models]{Хавсралт Б}-д оруулсан загварууд нь өгөгдлийн бүтцийг тодорхойлж, 360 хэмжээний санал хүсэлтийн нийлмэл өгөгдлийг удирдахад тусална.
    \item \textbf{Front-end (Next.js, Tailwind CSS)}: Хэрэглэгчдэд ээлтэй интерфейсээр гүйцэтгэлийн тайлан, OKR хяналтын самбарыг харуулдаг. 
    \hyperref[lst:frontend]{Хавсралт В}-д оруулсан код нь хариу үйлдэлтэй KPI самбарыг бий болгож, байгууллагуудын брэндийн онцлогт тохируулах боломжийг олгоно.
\end{itemize}
Эдгээр кодын хэсгүүд нь EPES системийн бодит цагийн хяналт, хэрэглэгчийн туршлага, өгөгдлийн найдвартай удирдлагын шаардлагыг хангаж, хөгжүүлэлтийн 
үр дүнд гүйцэтгэлийн үнэлгээний автоматжуулсан, хямд, local хэрэглээнд тохирсон системийг бий болгоход хувь нэмэр оруулсан.

\subsection{Тулгарсан асуудлууд ба шийдэл}
Хэрэгжүүлэлтийн явцад дараах асуудлууд тулгарч, тэдгээрийг доорх аргаар шийдвэрлэсэн:
\begin{itemize}
    \item \textbf{Асуудал 1: API удаашрал}: Олон хэрэглэгчийн зэрэг хүсэлтийг боловсруулахад Gin-ийн API-ийн хариу өгөх хугацаа удааширсан. \\
    \textbf{Шийдэл}: PostgreSQL-ийн индексжүүлэлчийг оновчтой болгож, Gin-ийн middleware-д cache хийх логик нэмсэн. Жишээлбэл, KPI тайлангийн асуулгад Redis cache-ийг нэвтрүүлж, 
    хариу өгөх хугацааг 40\%-аар бууруулсан.
    \item \textbf{Асуудал 2: Front-end хариу үйлдлийн удаашрал}: Том хэмжээний гүйцэтгэлийн өгөгдлийг харуулахад Next.js-ийн rendering удааширсан. \\
    \textbf{Шийдэл}: Next.js-ийн Incremental Static Regeneration (ISR)-ийг ашиглан статик хуудсуудыг урьдчилан үүсгэж, Tailwind CSS-ийн utility классуудыг тодорхой болгосон. 
    Энэ нь хуудасны ачааллын хугацааг 30\%-аар хурдасгасан.
    \item \textbf{Асуудал 3: Docker контейнерын нөөцийн хэт хэрэглээ}: Хөгжүүлэлтийн орчинд Docker контейнеруудын RAM болон CPU-ийн хэрэглээ өндөр байсан. \\
    \textbf{Шийдэл}: Docker Compose тохиргоог оновчтой болгож, контейнерын нөөцийн хязгаарлалтыг тогтоосон. Мөн PostgreSQL-ийн тохиргоонд хэт ачааллыг бууруулахын 
    тулд connection pooling нэмсэн.
\end{itemize}

\subsection{Хэрэгжүүлэлтийн үр дүнгийн дүгнэлт}
EPES системийн хэрэгжүүлэлт амжилттай хийгдэж, бодит цагийн гүйцэтгэлийн хяналт, 360 хэмжээний санал хүсэлт, OKR болон KPI-д суурилсан үнэлгээний модулиудыг багтаасан вэбд 
суурилсан систем бий болсон. Golang болон Next.js-ийн хослол нь хурдан, хэрэглэгчдэд ээлтэй платформыг хангасан бол PostgreSQL болон Docker нь өгөгдлийн найдвартай удирдлага, 
нэвтрүүлэлтийн тогтвортой байдлыг баталгаажуулсан. Тулгарсан асуудлуудыг оновчтой шийдвэрлэснээр системийн гүйцэтгэл, хүртээмжийг сайжруулж, жижиг, дунд байгууллагуудад 
хямд, local хэрэглээнд тохирсон шийдэл санал болгох боломжтой болсон. Гэсэн хэдий ч системийн том хэмжээний туршилт, аналитикийн модулийн өргөтгөл зэрэг нь цаашдын хөгжүүлэлтийн 
чухал чиглэл болно.



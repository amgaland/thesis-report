%!TEX TS-program = xelatex
% !TeX program = xelatex
%!TEX encoding = UTF-8 Unicode
%----------------------------------------------------------------------------------------
%   Доорх хэсгийг өөрчлөх шаардлагагүй
%----------------------------------------------------------------------------------------
\documentclass[12pt,A4]{report}

\usepackage{fontspec,xltxtra,xunicode}
\setmainfont[Ligatures=TeX]{Times New Roman}
\setsansfont{Arial}

% \usepackage[utf8x]{inputenc}
% \usepackage[mongolian]{babel}
%\usepackage{natbib}
\usepackage{geometry}
%\usepackage{fancyheadings} fancyheadings is obsolete: replaced by fancyhdr. JL
\usepackage{fancyhdr}
\usepackage{float}
\usepackage{afterpage}
\usepackage{graphicx}
\usepackage{amsmath,amssymb,amsbsy}
\usepackage{dcolumn,array}
\usepackage{tocloft}
\usepackage{dics}
\usepackage{nomencl}
\usepackage{upgreek}
\newcommand{\argmin}{\arg\!\min}
\usepackage{mathtools}
\usepackage[hidelinks]{hyperref}

\usepackage{algorithm}
\usepackage{algpseudocode}

\usepackage{listings}
\DeclarePairedDelimiter\abs{\lvert}{\rvert}%
\makeatletter
\usepackage{caption}
\captionsetup[table]{belowskip=0.5pt}
\captionsetup[table]{name=Хүснэгт}
\usepackage{subfiles}

\usepackage{listings}

\usepackage{color}
\definecolor{lightgray}{rgb}{.9,.9,.9}
\definecolor{darkgray}{rgb}{.4,.4,.4}
\definecolor{purple}{rgb}{0.65, 0.12, 0.82}

\lstdefinelanguage{TypeScript}{
  keywords={abstract, any, as, boolean, break, case, catch, class, console, 
    const, continue, debugger, declare, default, delete, do, else, enum, export, 
    extends, false, finally, for, from, function, get, if, implements, import, in, 
    infer, instanceof, interface, keyof, let, module, namespace, never, new, null, 
    number, object, package, private, protected, public, readonly, require, return, 
    set, static, string, super, switch, symbol, this, throw, true, try, type, typeof, 
    undefined, unique, unknown, var, void, while, with, yield, async, await},
  keywordstyle=\color{blue}\bfseries,
  ndkeywords={class, export, boolean, throw, implements, import, this},
  ndkeywordstyle=\color{darkgray}\bfseries,
  identifierstyle=\color{black},
  sensitive=false,
  comment=[l]{//},
  morecomment=[s]{/*}{*/},
  commentstyle=\color{purple}\ttfamily,
  stringstyle=\color{red}\ttfamily,
  morestring=[b]',
  morestring=[b]"
}

\lstset{
   language=TypeScript,
   backgroundcolor=\color{lightgray},
   extendedchars=true,
   basicstyle=\footnotesize\ttfamily,
   showstringspaces=false,
   showspaces=false,
   numbers=left,
   numberstyle=\footnotesize,
   numbersep=9pt,
   tabsize=2,
   breaklines=true,
   showtabs=false,
   captionpos=b
}
\lstdefinelanguage{yaml}{
  keywords={true,false,null,y,n},
  keywordstyle=\color{darkgray}\bfseries,
  basicstyle=\ttfamily\footnotesize,
  sensitive=false,
  comment=[l]{\#},
  morecomment=[s]{/*}{*/},
  commentstyle=\color{purple}\ttfamily,
  stringstyle=\color{blue}\ttfamily,
  morestring=[b]',
  morestring=[b]",
  breaklines=true,
  breakatwhitespace=true
}
\renewcommand{\lstlistingname}{Код}
\renewcommand{\lstlistlistingname}{\lstlistingname ын жагсаалт}

\usepackage{color}
\definecolor{codegreen}{rgb}{0,0.6,0}
\definecolor{codegray}{rgb}{0.5,0.5,0.5}
\definecolor{codepurple}{rgb}{0.58,0,0.82}
\definecolor{backcolour}{rgb}{0.99,0.99,0.99}
 
\lstdefinestyle{mystyle}{
    basicstyle=\ttfamily\small,
    backgroundcolor=\color{backcolour},   
    commentstyle=\color{codegreen},
    keywordstyle=\color{magenta},
    numberstyle=\tiny\color{codegray},
    stringstyle=\color{codepurple},
    %basicstyle=\footnotesize,
    breakatwhitespace=false,         
    breaklines=true,                 
    captionpos=b,                    
    keepspaces=false,                 
    numbers=left,                    
    numbersep=10pt,                  
    showspaces=false,                
    showstringspaces=true,
    showtabs=false,                  
    tabsize=2
}
 
\lstset{style=mystyle, label=DescriptiveLabel} 

\let\oldabs\abs
\def\abs{\@ifstar{\oldabs}{\oldabs*}}
\makenomenclature
\begin{document}


%----------------------------------------------------------------------------------------
%   Өөрийн мэдээллээ оруулах хэсэг
%----------------------------------------------------------------------------------------

% Дипломийн ажлын сэдэв
\title{Бизнесийн байгууллагын ажилтаны гүйцэтгэлийг үнэлэх систем}
% Дипломын ажлын англи нэр
\titleEng{Employee performance evaluation system for business organization}
% Өөрийн овог нэрийг бүтнээр нь бичнэ
\author{Баянжаргалын Энх-Амгалан}
% Өөрийн овгийн эхний үсэг нэрээ бичнэ
\authorShort{Б. Энх-Амгалан}
% Удирдагчийн зэрэг цол овгийн эхний үсэг нэр
\supervisor{Б. Энхтуул}
% Хамтарсан удирдагчийн зэрэг цол овгийн эхний үсэг нэр
% \cosupervisor{Н. Оюун-Эрдэнэ}

% СиСи дугаар 
\sisiId{21B1NUM0344}
% Их сургуулийн нэр
\university{МОНГОЛ УЛСЫН ИХ СУРГУУЛЬ}
% Бүрэлдэхүүн сургуулийн нэр
\faculty{МЭДЭЭЛЛИЙН ТЕХНОЛОГИ, ЭЛЕКТРОНИКИЙН СУРГУУЛЬ}
% Тэнхимийн нэр
\department{МЭДЭЭЛЭЛ, КОМПЬЮТЕРЫН УХААНЫ ТЭНХИМ}
% Зэргийн нэр
\degreeName{Бакалаврын судалгааны ажил}
% Суралцаж буй хөтөлбөрийн нэр
\programeName{Мэдээллийн технологи(D061303)}
% Хэвлэгдсэн газар
\cityName{Улаанбаатар}
% Хэвлэгдсэн огноо
\gradyear{2025 оны 5 сар}


%----------------------------------------------------------------------------------------
%   Доорх хэсгийг өөрчлөх шаардлагагүй
%----------------------------------------------------------------------------------------
\include{src/important/main-pre}

% Удиртгалыг оруулж ирэх ба abstract.tex файлд удиртгалаа бичнэ
\begin{abstract}
	\setcounter{secnumdepth}{0}
	
	Бизнесийн байгууллагуудын өрсөлдөх чадвар, амжилт нь ажилтнуудын гүйцэтгэлээс ихээхэн хамаардаг. 
    Ажилтны гүйцэтгэлийг үнэлэх нь байгуулла зорилгодоо хүрэх, бүтээмжийг нэмэгдүүлэхэд чухал үүрэгтэй. 
    Энэхүү судалгааны ажлын зорилго нь бизнесийн байгууллагад зориулсан ажилтны гүйцэтгэлийн үнэлгээний системийг вебд суурилан бүтээхэд оршино.

    Энэ хүрээнд Next.js болон Golang хэл дээр суурилсан веб апп-ийг хөгжүүлсэн бөгөөд уг систем нь төслийн удирдлага, 
    даалгаврын менежмент, ажилтны гүйцэтгэлийн үнэлгээний систем зэргийг нэгтгэсэн болно. Тус систем нь удирдлага болон 
    ажилтнуудын хамтын ажиллагааг дэмжиж, даалгаврын хуваарилалт, гүйцэтгэлийн хяналт, үнэлгээний процессыг автоматжуулан, 
    илүү үр дүнтэй, шударга системийг бий болгохыг зорьдог. Судалгаагаар энэхүү системийн онолын загвар, хэрэгжилт, 
    удирдлагын арга барилд үзүүлэх нөлөөг авч үзнэ.
    
    Энэхүү ажлын үр дүнд бизнесийн байгууллагын удирдах албан тушаалтан болон хүний нөөцийн мэргэжилтнүүдэд ажилтны чадавхийг нээн илрүүлэх, 
    гүйцэтгэлийг дээшлүүлэхэд чиглэсэн шийдвэр гаргалтанд дэмжлэг үзүүлэхэд технологийн дэвшилтийг ашиглахад оршино.

	%	\setcounter{secnumdepth}{0} reverse this command
	\setcounter{secnumdepth}{2}

\end{abstract}
\addcontentsline{toc}{part}{БҮЛГҮҮД}

%----------------------------------------------------------------------------------------
%   Дипломын үндсэн хэсэг эндээс эхэлнэ
%----------------------------------------------------------------------------------------
% Шинэ бүлэг
\chapter{СИСТЕМИЙН ТАНИЛЦУУЛГА}
\subfile{src/chapters/chapter1/chapter1}

\chapter{СИСТЕМИЙН СУДАЛГАА}
\subfile{src/chapters/chapter2/chapter2}

\chapter{СИСТЕМИЙН ШИНЖИЛГЭЭ}
\subfile{src/chapters/chapter3/chapter3}

% \chapter{СИСТЕМИЙН ЗОХИОМЖ}
% \subfile{src/chapters/chapter4/chapter4}

% \chapter{ХЭРЭГЖҮҮЛЭЛТ}
% \subfile{src/chapters/chapter5/chapter5}

% \chapter{ДҮГНЭЛТ}





% \chapter{Тайлан боловсруулах зөвлөмж}
% \subfile{writing.tex}


%----------------------------------------------------------------------------------------
%   Дүгнэлт эндээс эхэлнэ
%----------------------------------------------------------------------------------------

%----------------------------------------------------------------------------------------
%   Дипломын номзүй, хавсралтын хэсэг эндээс эхэлнэ
%----------------------------------------------------------------------------------------

\singlespace
\addcontentsline{toc}{part}{НОМ ЗҮЙ}
\begin{thebibliography}{}
	% Ашигласан материалыг эндээс оруулна
	\bibitem{Islam2006}
	R. Islam and S. M. Rasad, ``Employee performance evaluation by the AHP: A case study,'' \emph{Asia Pacific Management Review}, vol. 11, no. 3, pp. 163--176, Jun. 2006. [Online]. Available: \url{https://rafikulislam.com/uploads/myworks/27066075955b8a1b374434.pdf}

	% Reference for Performance Evaluation Methods and Techniques
	\bibitem{Shaout2014}
	A. Shaout and M. K. Yousif, ``Performance evaluation – Methods and techniques survey,'' \emph{MCST Journal}, vol. 3, no. 5, pp. 66--74, Sep. 2014. [Online]. Available: \url{https://d1wqtxts1xzle7.cloudfront.net/41797022/Paper030516-libre.pdf}

	% Reference for Next.js documentation
	\bibitem{NextJS2025}
	Next.js Team, ``Next.js documentation,'' Vercel Inc., 2025. [Online]. Available: \url{https://nextjs.org/docs}

	% Reference for Golang documentation
	\bibitem{Go2025}
	The Go Authors, ``The Go programming language documentation,'' 2025. [Online]. Available: \url{https://go.dev/doc/}

	% Reference for PostgreSQL documentation
	\bibitem{PostgreSQL2025}
	PostgreSQL Global Development Group, ``PostgreSQL documentation,'' 2025. [Online]. Available: \url{https://www.postgresql.org/docs/}

	% Reference for Docker documentation
	\bibitem{Docker2025}
	Docker Inc., ``Docker documentation,'' 2025. [Online]. Available: \url{https://docs.docker.com/}

	% Reference for Performance Management book
	\bibitem{Armstrong2015}
	M. Armstrong, \emph{Performance Management: Key Strategies and Practical Guidelines}, 5th ed. London, UK: Kogan Page, 2015.

	% Reference for Computer Networks book
	\bibitem{Tanenbaum2021}
	A. S. Tanenbaum and D. J. Wetherall, \emph{Computer Networks}, 6th ed. Upper Saddle River, NJ: Pearson, 2021.

\end{thebibliography}


%----------------------------------------------------------------------------------------
%   Хавсралтууд эндээс эхэлнэ
%----------------------------------------------------------------------------------------
\appendix
\addcontentsline{toc}{part}{ХАВСРАЛТ}

% Хавсралтын нэр. Хавсралт гэдэг үг агуулахгүй
% \chapter{Шинжилгээ зохиомж}
% Хавсралтын агуулга

% Хавсралтын нэр. Хавсралт гэдэг үг агуулахгүй
% \chapter{Кодын хэрэгжүүлэлт}

% \lstinputlisting[language=Typescript, caption=tRPC тохиргоо,label=lst:trpc,frame=single]{src/code/trpc.ts}

% \lstinputlisting[language=yaml, caption=Docker Compose ,label=lst:docker-compose,frame=single]{src/code/docker-compose.yml}

\chapter{REST API-ийн жишээ код (Golang, Gin)}
\label{lst:api}
\begin{lstlisting}[language=Golang, caption=Routes, frame=single]
package main

import (
    "github.com/gin-gonic/gin"
    "net/http"
)

// KPI мэдээлэл авах endpoint
func getKPI(c *gin.Context)point {
    employeeID := c.Param("id")
    var kpi []KPIData
    // PostgreSQL-аас KPI мэдээлэл татах
    if err := db.Where("employee_id = ?", employeeID).Find(&kpi).Error; err != nil {
        c.JSON(http.StatusInternalServerError, gin.H{"error": "Failed to fetch KPI"})
        return
    }
    c.JSON(http.StatusOK, kpi)
}

func main() {
    router := gin.Default()
    router.GET("/kpi/:id", getKPI)
    router.Run(":8080")
}
\end{lstlisting}

\chapter{Back-End хэрэгжүүлэлт}

\par Өгөгдлийн санд байрлах бүхий л өгөгдлүүд рүү хандах API-ууд route-үүд
\begin{lstlisting}[language=Golang, caption=Routes, frame=single]
package routes

import (
	"github.com/amgaland/epes/epes-back/controllers"
	admin "github.com/amgaland/epes/epes-back/controllers/admin"
	api "github.com/amgaland/epes/epes-back/controllers/api"
	protected "github.com/amgaland/epes/epes-back/controllers/protected"
	"github.com/gin-gonic/gin"
)

func RegisterRoutes(router *gin.Engine) {
    router.GET("/health", controllers.HealthCheck)

    authRoutes := router.Group("/auth")
    {
        authRoutes.POST("/signin", api.SignIn)
    }

    adminRoutes := router.Group("/admin")
    {
        userRoutes := adminRoutes.Group("/users")
        {
            userRoutes.GET("/", admin.GetAllUsers)
            userRoutes.POST("/", admin.CreateUser)
            userRoutes.PUT("/:id", admin.UpdateUser)
            userRoutes.DELETE("/:id", admin.DeleteUser)
            userRoutes.GET("/check-login-id", admin.CheckLoginIDExists)
        }

        userRoleRoutes := adminRoutes.Group("/user/roles")
        {
            userRoleRoutes.GET("/", admin.GetAllUserRoles)
            userRoleRoutes.POST("/", admin.CreateUserRole)
            userRoleRoutes.PUT("/:id", admin.UpdateUserRole)
            userRoleRoutes.DELETE("/:id", admin.DeleteUserRole)
            userRoleRoutes.GET("/list", admin.UserRoleHandler)
            userRoleRoutes.PUT("/update", admin.UpdateUserRoleHandler)
        }

        roleRoutes := adminRoutes.Group("/roles")
        {
            roleRoutes.GET("/", admin.GetAllRoles)
            roleRoutes.POST("/", admin.CreateRole)
            roleRoutes.PUT("/:id", admin.UpdateRole)
            roleRoutes.DELETE("/:id", admin.DeleteRole)
        }

        rolePermissionRoutes := adminRoutes.Group("/role-permissions")
        {
            rolePermissionRoutes.GET("/", admin.GetAllRolePermissions)
            rolePermissionRoutes.POST("/", admin.CreateRolePermission)
            rolePermissionRoutes.PUT("/:id", admin.UpdateRolePermission)
            rolePermissionRoutes.DELETE("/:id", admin.DeleteRolePermission)
            rolePermissionRoutes.GET("/list", admin.RolePermissionHandler)
            rolePermissionRoutes.PUT("/update", admin.UpdateRolePermissionHandler)
        }

        actionTypeRoutes := adminRoutes.Group("/action-types")
        {
            actionTypeRoutes.GET("/", admin.GetAllActionTypes)
            actionTypeRoutes.POST("/", admin.CreateActionType)
            actionTypeRoutes.PUT("/:id", admin.UpdateActionType)
            actionTypeRoutes.DELETE("/:id", admin.DeleteActionType)
        }
    }

    protectedRoutes := router.Group("/protected")
    {
        departmentRoutes := protectedRoutes.Group("/departments")
        {
            departmentRoutes.GET("/", protected.GetAllDepartments)
            departmentRoutes.POST("/", protected.CreateDepartment)
            departmentRoutes.PUT("/:id", protected.UpdateDepartment)
            departmentRoutes.DELETE("/:id", protected.DeleteDepartment)
        }

        userDepartmentRoutes := protectedRoutes.Group("/user/departments")
        {
            userDepartmentRoutes.GET("/", protected.GetAllUserDepartments)
            userDepartmentRoutes.POST("/", protected.CreateUserDepartment)
            userDepartmentRoutes.PUT("/:id", protected.UpdateUserDepartment)
            userDepartmentRoutes.DELETE("/:id", protected.DeleteUserDepartment)
            userDepartmentRoutes.GET("/list", protected.UserDepartmentHandler)
            userDepartmentRoutes.PUT("/update", protected.UpdateUserDepartmentHandler)
        }
        taskRoutes := protectedRoutes.Group("/tasks")
        {
            taskRoutes.GET("/", protected.GetAllTasks)
            taskRoutes.POST("/", protected.CreateTask)
            taskRoutes.PUT("/:id", protected.UpdateTask)
            taskRoutes.DELETE("/:id", protected.DeleteTask)
            taskRoutes.GET("/check-task-id", protected.CheckTaskIDExists)
        }
        projectRoutes := protectedRoutes.Group("/projects")
        {
            projectRoutes.GET("/", protected.GetAllProjects)
            projectRoutes.POST("/", protected.CreateProject)
            projectRoutes.PUT("/:id", protected.UpdateProject)
            projectRoutes.DELETE("/:id", protected.DeleteProject)
            projectRoutes.GET("/tasks/:id", protected.GetAllProjectTasks)
        }

        projectMemberRoutes := projectRoutes.Group("/members")
        {
            projectMemberRoutes.GET("/project/member", protected.GetProjectMember)
            projectMemberRoutes.POST("/project/member", protected.CreateProjectMember)
            projectMemberRoutes.DELETE("/project/member/:id", protected.DeleteProjectMember)

        }

        kpiRoutes := protectedRoutes.Group("/kpi")
        {
            kpiRoutes.POST("/", protected.CreateKPI)
            kpiRoutes.POST("/employee-kpi", protected.CreateEmployeeKPI)
            kpiRoutes.POST("/employee-kpi/:id", protected.CreateEmployeeKPI)
            kpiRoutes.GET("/", protected.GetEmployeeKPIs)
            kpiRoutes.GET("/employee-kpi/:id", protected.GetEmployeeKPIByID)
            kpiRoutes.DELETE("/employee-kpi/:id", protected.DeleteEmployeeKPI)
            kpiRoutes.PUT("/employee-kpi/:id", protected.UpdateEmployeeKPI)
        }
        employeeRoutes := protectedRoutes.Group("/employees")
        {
            employeeRoutes.GET("/", protected.GetAllEmployees)
            employeeRoutes.POST("/", protected.CreateEmployee)
            employeeRoutes.PUT("/:id", protected.UpdateEmployee)
            employeeRoutes.DELETE("/:id", protected.DeleteEmployee)
            employeeRoutes.GET("/:id", protected.GetEmployeeByID)
            
        }
        feedbackRoutes := protectedRoutes.Group("/feedback")
        {
            feedbackRoutes.GET("/", protected.GetAllFeedback)
            feedbackRoutes.POST("/", protected.CreateFeedback)
            feedbackRoutes.PUT("/:id", protected.UpdateFeedback)
            feedbackRoutes.DELETE("/:id", protected.DeleteFeedback)
            feedbackRoutes.GET("/:id", protected.GetFeedbackByID)
            
        }
    }

}
\end{lstlisting}

\par Back-End моделүүд
\begin{lstlisting}[language=Golang, caption=Models, frame=single]

type Model struct {
    ID        string    `json:"id" gorm:"default:gen_random_uuid()"`
    CreatedAt time.Time `json:"created_at" gorm:"default:now()"`
    UpdatedAt time.Time `json:"updated_at" gorm:"default:now()"`
    CreatedBy *string   `json:"created_by" gorm:"references:User:ID"`
    UpdatedBy *string   `json:"updated_by" gorm:"references:User:ID"`
}
    

type User struct {
	Model   
	FirstName           string        `json:"first_name"`
	LastName            string        `json:"last_name"`
	LoginID             string        `json:"login_id" gorm:"unique"`
	EmailWork           string        `json:"email_work"`
	EmailPersonal       *string       `json:"email_personal"`
	PhoneNumberWork     *string       `json:"phone_number_work"`
	PhoneNumberPersonal *string       `json:"phone_number_personal"`
	IsActive            *bool         `json:"is_active"`
	ActiveStartDate     time.Time     `json:"active_start_date"`
	ActiveEndDate       *time.Time    `json:"active_end_date"`
	Password            string        `json:"password"`
}

type LoginUser struct {
	LoginID  string `json:"login_id"`
	Password string `json:"password"`
}

type UserWithRoles struct {
	ID                  string     `json:"id"`
	FirstName           string     `json:"first_name"`
	LastName            string     `json:"last_name"`
	EmailPersonal       string     `json:"email_personal"`
	EmailWork           string     `json:"email_work"`
	LoginID             string     `json:"login_id"`
	PhoneNumberPersonal string     `json:"phone_number_personal"`
	PhoneNumberWork     string     `json:"phone_number_work"`
	IsActive            bool       `json:"is_active"`
	Token               string     `json:"token"`
	ActiveStartDate     time.Time  `json:"active_start_date"`
	ActiveEndDate       *time.Time `json:"active_end_date,omitempty"`
	Roles               string     `json:"roles"`
}

type Role struct {
	Model
	Name string `json:"name" gorm:"unique"`
}

type Project struct {
	Model
	Name           string          `json:"name"`
	Description    string          `json:"description"`
	StartDate      time.Time       `json:"start_date"`
	EndDate        *time.Time      `json:"end_date"`
	Status         string          `json:"status"`        // e.g., "Ongoing", "Completed", "Delayed"
	OwnerID        string          `json:"owner_id"`      // Reference to User
	Owner          User            `json:"owner" gorm:"foreignKey:OwnerID;references:ID"`
	// Relationship to team members
	TeamMembers    []ProjectMember `json:"team_members" gorm:"foreignKey:ProjectID"`
}

type ProjectMember struct {
	Model
	ProjectID      string   `json:"project_id"`
	Project        Project  `json:"project" gorm:"foreignKey:ProjectID;references:ID"`
	UserID         string   `json:"user_id"`
	User           User     `json:"user" gorm:"foreignKey:UserID;references:ID"`
	RoleInProject  string   `json:"role_in_project"` // Optional: "Manager", "Developer", "QA", etc.
}

type EmployeeEvaluationReport struct {
	Model
	EmployeeID  string `json:"employee_id"`
	Employee    Employee `gorm:"foreignKey:EmployeeID"`
	Period      string `json:"period"`   // e.g., "Q1 2025"
	ReportData  string `json:"report_data"` // JSON or HTML summary
	GeneratedBy string `json:"generated_by"`
	GeneratedAt time.Time `json:"generated_at"`
}
type Task struct {
	Model
	ProjectID      string     `json:"project_id"`
	Project        Project    `json:"project" gorm:"foreignKey:ProjectID;references:ID"`
	Title          string     `json:"title"`
	Description    string     `json:"description"`
	AssignedToID   string     `json:"assigned_to_id"`
	AssignedTo     User       `json:"assigned_to" gorm:"foreignKey:AssignedToID;references:ID"`
	Status         string     `json:"status"`         // e.g., "Pending", "In Progress", "Completed"
	Deadline       *time.Time `json:"deadline"`
	CompletedAt    *time.Time `json:"completed_at"`
}

type TaskFeedback struct {
	Model
	TaskID      string `json:"task_id"`
	Task        Task   `json:"task" gorm:"foreignKey:TaskID;references:ID"`
	EvaluatorID string `json:"evaluator_id"`
	Evaluator   User   `json:"evaluator" gorm:"foreignKey:EvaluatorID;references:ID"`
	Comment     string `json:"comment"`
	Rating      int    `json:"rating"`  // 1-5 or percentage
}


\end{lstlisting}

\chapter{Front-End жишээ код (Next.js)}
\label{lst:api}
\begin{lstlisting}[language=Golang, caption=Routes, frame=single]
    "use client";

    import { useState, useEffect, useMemo, useCallback } from "react";
    import { useSession } from "next-auth/react";
    import { useRouter } from "next/navigation";
    import { useToast } from "@/hooks/use-toast";
    import { Button } from "@/components/ui/button";
    import { Input } from "@/components/ui/input";
    import { Skeleton } from "@/components/ui/skeleton";
    import { Card, CardContent, CardHeader, CardTitle } from "@/components/ui/card";
    import { Separator } from "@/components/ui/separator";
    import {
      BarChart,
      CheckCircle,
      CirclePlus,
      Clock,
      Download,
      LayoutGrid,
      List,
      Table as TableIcon,
      Users,
      RefreshCw,
    } from "lucide-react";
    import { Search } from "lucide-react";
    import { KPIStats } from "./components/KPIStats";
    import { KPIActions } from "./components/KPIActions";
    import { KPITable } from "./components/KPITable";
    import { KPIGrid } from "./components/KPIGrid";
    import { KPIList } from "./components/KPIList";
    import { KPIReportDialog } from "./components/KPIReportDialog";
    import { DeleteDialog } from "./components/DeleteDialog";
    import { fetchEmployeeKPIs, deleteKPI } from "./services/kpiService";
    import {
      sortKPIs,
      filterKPIs,
      exportToCSV,
      generatePerformanceReport,
    } from "./utils/kpiUtils";
    import { EmployeeKPI, KPIStat, ReportConfig } from "./types";
    
    // Custom debounce function to avoid lodash dependency
    const debounce = <F extends (...args: any[]) => void>(
      func: F,
      wait: number
    ) => {
      let timeout: NodeJS.Timeout;
      return (...args: Parameters<F>) => {
        clearTimeout(timeout);
        timeout = setTimeout(() => func(...args), wait);
      };
    };
    
    const KPIPage: React.FC = () => {
      const { data: session, status } = useSession();
      const router = useRouter();
      const { toast } = useToast();
      const [isLoading, setIsLoading] = useState(true);
      const [viewMode, setViewMode] = useState<"table" | "grid" | "list">("table");
      const [searchTerm, setSearchTerm] = useState("");
      const [searchInput, setSearchInput] = useState("");
      const [kpis, setKPIs] = useState<EmployeeKPI[]>([]);
      const [stats, setStats] = useState<KPIStat[]>([]);
      const [sortField, setSortField] = useState<keyof EmployeeKPI | null>(null);
      const [sortDirection, setSortDirection] = useState<"asc" | "desc">("asc");
      const [filterStatus, setFilterStatus] = useState<
        "All" | "Excellent" | "Good" | "Needs Improvement"
      >("All");
      const [deleteDialogOpen, setDeleteDialogOpen] = useState(false);
      const [employeeToDelete, setEmployeeToDelete] = useState<string | null>(null);
      const [reportDialogOpen, setReportDialogOpen] = useState(false);
      const [reportConfig, setReportConfig] = useState<ReportConfig>({
        employeeId: "all",
        period: "allTime",
        includeTasks: true,
        includeProjects: true,
        includeComments: false,
      });
    
      // Centralized KPI metrics calculation
      const calculateKPIMetrics = useCallback((kpiData: EmployeeKPI[]) => {
        const excellentPerformers = kpiData.filter(
          (k) => k.status === "Excellent"
        ).length;
        const goodPerformers = kpiData.filter((k) => k.status === "Good").length;
        const needsImprovement = kpiData.filter(
          (k) => k.status === "Needs Improvement"
        ).length;
        const avgPerformanceScore =
          kpiData.length > 0
            ? Math.round(
                kpiData.reduce((sum, k) => sum + k.performanceScore, 0) /
                  kpiData.length
              )
            : 0;
        const totalEmployees = kpiData.length;
    
        return {
          stats: [
            {
              title: "Excellent Performers",
              value: excellentPerformers,
              icon: CheckCircle,
            },
            { title: "Good Performers", value: goodPerformers, icon: Users },
            { title: "Needs Improvement", value: needsImprovement, icon: Clock },
            {
              title: "Avg Performance Score",
              value: avgPerformanceScore,
              icon: BarChart,
            },
          ],
          metrics: {
            totalEmployees,
            excellentPerformers,
            avgPerformanceScore,
          },
        };
      }, []);
    
      // Debounced search handler
      const debouncedSearch = useMemo(
        () =>
          debounce((value: string) => {
            setSearchTerm(value);
          }, 300),
        []
      );
    
      // Handle search input change
      const handleSearchChange = useCallback(
        (e: React.ChangeEvent<HTMLInputElement>) => {
          const value = e.target.value;
          setSearchInput(value);
          debouncedSearch(value);
        },
        [debouncedSearch]
      );
    
      // Reset all filters
      const resetFilters = useCallback(() => {
        setSearchInput("");
        setSearchTerm("");
        setFilterStatus("All");
        setSortField(null);
        setSortDirection("asc");
        toast({
          title: "Filters Reset",
          description: "All filters have been cleared.",
        });
      }, [toast]);
    
      useEffect(() => {
        const fetchData = async () => {
          if (!session?.user?.token) {
            toast({
              title: "Authentication Error",
              description: "Authentication token missing. Please log in again.",
              variant: "destructive",
            });
            router.push("/auth/signin");
            return;
          }
    
          try {
            setIsLoading(true);
            const kpiData = await fetchEmployeeKPIs(session.user.token);
            setKPIs(kpiData);
    
            const { stats } = calculateKPIMetrics(kpiData);
            setStats(stats);
          } catch (error: any) {
            console.error("Failed to fetch KPIs:", error);
            toast({
              title: "Error",
              description: "Failed to load KPIs: " + error.message,
              variant: "destructive",
            });
            setKPIs([]);
            setStats([
              { title: "Excellent Performers", value: 0, icon: CheckCircle },
              { title: "Good Performers", value: 0, icon: Users },
              { title: "Needs Improvement", value: 0, icon: Clock },
              { title: "Avg Performance Score", value: 0, icon: BarChart },
            ]);
          } finally {
            setIsLoading(false);
          }
        };
    
        if (session) {
          fetchData();
        }
      }, [session, router, toast, debouncedSearch, calculateKPIMetrics]);
    
      const roles = session?.user?.roles
        ? Array.isArray(session.user.roles)
          ? session.user.roles
          : [session.user.roles]
        : [];
      const isAdmin = roles.includes("ADMIN");
    
      const handleSort = useCallback(
        (field: keyof EmployeeKPI) => {
          if (sortField === field) {
            setSortDirection(sortDirection === "asc" ? "desc" : "asc");
          } else {
            setSortField(field);
            setSortDirection("asc");
          }
        },
        [sortField, sortDirection]
      );
    
      const handleKPIClick = useCallback(
        (employeeId: string) => {
          router.push(`/protected/kpi/employee/${employeeId}`);
        },
        [router]
      );
    
      const handleEditKPI = useCallback(
        (employeeId: string) => {
          router.push(`/protected/kpi/edit/${employeeId}`);
        },
        [router]
      );
    
      const handleDeleteKPI = useCallback(async () => {
        if (!employeeToDelete || !session?.user?.token) return;
    
        try {
          await deleteKPI(employeeToDelete, session.user.token);
          setKPIs((prev) =>
            prev.filter((kpi) => kpi.employeeId !== employeeToDelete)
          );
          const { stats } = calculateKPIMetrics(
            kpis.filter((kpi) => kpi.employeeId !== employeeToDelete)
          );
          setStats(stats);
          toast({
            title: "Success",
            description: "KPI record deleted successfully.",
          });
        } catch (error: any) {
          console.error("Failed to delete KPI:", error);
          toast({
            title: "Error",
            description: "Failed to delete KPI: " + error.message,
            variant: "destructive",
          });
        } finally {
          setDeleteDialogOpen(false);
          setEmployeeToDelete(null);
        }
      }, [employeeToDelete, session, toast, kpis, calculateKPIMetrics]);
    
      const sortedKPIs = useMemo(
        () => sortKPIs(kpis, sortField, sortDirection),
        [kpis, sortField, sortDirection]
      );
      const filteredKPIs = useMemo(
        () => filterKPIs(sortedKPIs, searchTerm, filterStatus),
        [sortedKPIs, searchTerm, filterStatus]
      );
    
      // Calculate metrics for Quick Info
      const { metrics } = useMemo(
        () => calculateKPIMetrics(kpis),
        [kpis, calculateKPIMetrics]
      );
    
      if (status === "loading") {
        return (
          <div className="flex min-h-screen bg-background">
            <div className="flex-1 flex flex-col">
              <main className="p-4 sm:p-6 flex-1">
                <Skeleton className="h-8 w-[200px] mb-6" />
                <KPIStats stats={[]} isLoading={true} />
              </main>
            </div>
          </div>
        );
      }
    
      if (status === "unauthenticated" || !session) {
        router.push("/auth/signin");
        return null;
      }
    
      if (!isAdmin) {
        toast({
          title: "Access Denied",
          description: "Only admins can access this page.",
          variant: "destructive",
        });
        router.push("/protected");
        return null;
      }
    
      return (
        <div className="flex min-h-screen bg-background">
          <div className="flex-1 flex flex-col max-w-7xl mx-auto w-full">
            {/* Search and Actions */}
            <Card className="sticky top-0 z-30 border-b border-border/50 bg-background/95 backdrop-blur supports-[backdrop-filter]:bg-background/60">
              <CardContent className="flex flex-col sm:flex-row items-center gap-4 p-4">
                <div className="relative w-full sm:w-auto flex-1">
                  <Search className="absolute left-2.5 top-2.5 h-4 w-4 text-muted-foreground" />
                  <Input
                    type="search"
                    value={searchInput}
                    onChange={handleSearchChange}
                    placeholder="Search employees..."
                    className="pl-8 w-full sm:w-[250px] lg:w-[350px]"
                  />
                </div>
                <div className="flex items-center gap-2 w-full sm:w-auto justify-between sm:justify-end">
                  <Button onClick={() => router.push("/protected/kpi/create")}>
                    <CirclePlus className="mr-2 h-4 w-4" />
                    Add KPI
                  </Button>
                  <Button
                    variant="outline"
                    onClick={() => exportToCSV(filteredKPIs)}
                    disabled={isLoading}
                  >
                    <Download className="mr-2 h-4 w-4" />
                    Export CSV
                  </Button>
                  <Button variant="outline" onClick={resetFilters}>
                    <RefreshCw className="mr-2 h-4 w-4" />
                    Reset
                  </Button>
                </div>
              </CardContent>
            </Card>
    
            <main className="p-4 sm:p-6 flex-1">
              <div className="flex justify-between items-center mb-6">
                <h1 className="text-2xl sm:text-3xl font-bold text-foreground">
                  Employee KPIs
                </h1>
                <div className="flex gap-2">
                  <Button
                    variant={viewMode === "table" ? "default" : "ghost"}
                    size="icon"
                    onClick={() => setViewMode("table")}
                  >
                    <TableIcon className="h-4 w-4" />
                  </Button>
                  <Button
                    variant={viewMode === "grid" ? "default" : "ghost"}
                    size="icon"
                    onClick={() => setViewMode("grid")}
                  >
                    <LayoutGrid className="h-4 w-4" />
                  </Button>
                  <Button
                    variant={viewMode === "list" ? "default" : "ghost"}
                    size="icon"
                    onClick={() => setViewMode("list")}
                  >
                    <List className="h-4 w-4" />
                  </Button>
                </div>
              </div>
    
              {/* Quick Filters */}
              <Card className="mb-6">
                <CardContent className="flex flex-wrap gap-2 p-4">
                  <Button
                    variant={filterStatus === "All" ? "default" : "outline"}
                    onClick={() => setFilterStatus("All")}
                  >
                    All
                  </Button>
                  <Button
                    variant={filterStatus === "Excellent" ? "default" : "outline"}
                    onClick={() => setFilterStatus("Excellent")}
                  >
                    Excellent
                  </Button>
                  <Button
                    variant={filterStatus === "Good" ? "default" : "outline"}
                    onClick={() => setFilterStatus("Good")}
                  >
                    Good
                  </Button>
                  <Button
                    variant={
                      filterStatus === "Needs Improvement" ? "default" : "outline"
                    }
                    onClick={() => setFilterStatus("Needs Improvement")}
                  >
                    Needs Improvement
                  </Button>
                </CardContent>
              </Card>
    
              <KPIStats stats={stats} isLoading={isLoading} />
    
              <div className="grid gap-4 md:grid-cols-2 mb-6">
                <Card>
                  <CardHeader>
                    <CardTitle>Actions</CardTitle>
                  </CardHeader>
                  <CardContent>
                    <KPIActions
                      isLoading={isLoading}
                      onCreate={() => router.push("/protected/kpi/create")}
                      onGenerateReport={() => setReportDialogOpen(true)}
                      onViewExcellent={() =>
                        router.push("/protected/kpi/excellent")
                      }
                      onViewAll={() => router.push("/protected/kpi/all")}
                    />
                  </CardContent>
                </Card>
                <Card>
                  <CardHeader>
                    <CardTitle>Quick Info</CardTitle>
                  </CardHeader>
                  <CardContent>
                    {isLoading ? (
                      <Skeleton className="h-20 w-full" />
                    ) : (
                      <div className="space-y-4">
                        <div className="flex justify-between items-center">
                          <span className="text-sm text-muted-foreground">
                            Total Employees
                          </span>
                          <span className="text-sm font-medium">
                            {metrics.totalEmployees}
                          </span>
                        </div>
                        <Separator />
                        <div className="flex justify-between items-center">
                          <span className="text-sm text-muted-foreground">
                            Excellent
                          </span>
                          <span className="text-sm font-medium">
                            {metrics.excellentPerformers}
                          </span>
                        </div>
                        <Separator />
                        <div className="flex justify-between items-center">
                          <span className="text-sm text-muted-foreground">
                            Avg Score
                          </span>
                          <span className="text-sm font-medium">
                            {metrics.avgPerformanceScore}
                          </span>
                        </div>
                      </div>
                    )}
                  </CardContent>
                </Card>
              </div>
    
              <Card>
                <CardHeader>
                  <CardTitle>Employee KPI List</CardTitle>
                </CardHeader>
                <CardContent>
                  {isLoading ? (
                    <div className="space-y-4">
                      <Skeleton className="h-8 w-full" />
                      <Skeleton className="h-32 w-full" />
                    </div>
                  ) : (
                    <>
                      {viewMode === "table" && (
                        <KPITable
                          kpis={filteredKPIs}
                          onSort={handleSort}
                          onClick={handleKPIClick}
                          onEdit={handleEditKPI}
                          onDelete={(id) => {
                            setEmployeeToDelete(id);
                            setDeleteDialogOpen(true);
                          }}
                        />
                      )}
                      {viewMode === "grid" && (
                        <KPIGrid
                          kpis={filteredKPIs}
                          onClick={handleKPIClick}
                          onEdit={handleEditKPI}
                          onDelete={(id) => {
                            setEmployeeToDelete(id);
                            setDeleteDialogOpen(true);
                          }}
                        />
                      )}
                      {viewMode === "list" && (
                        <KPIList
                          kpis={filteredKPIs}
                          onClick={handleKPIClick}
                          onEdit={handleEditKPI}
                          onDelete={(id) => {
                            setEmployeeToDelete(id);
                            setDeleteDialogOpen(true);
                          }}
                        />
                      )}
                    </>
                  )}
                </CardContent>
              </Card>
            </main>
    
            <DeleteDialog
              open={deleteDialogOpen}
              onOpenChange={setDeleteDialogOpen}
              onConfirm={handleDeleteKPI}
            />
    
            <KPIReportDialog
              open={reportDialogOpen}
              onOpenChange={setReportDialogOpen}
              kpis={kpis}
              config={reportConfig}
              setConfig={setReportConfig}
              onGenerate={() => {
                generatePerformanceReport(kpis, reportConfig);
                toast({
                  title: "Success",
                  description: "Performance report generated successfully as PDF.",
                });
                setReportDialogOpen(false);
              }}
            />
          </div>
        </div>
      );
    };
    
    export default KPIPage;
    

\end{lstlisting}

\end{document}
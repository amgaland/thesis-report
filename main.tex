%!TEX TS-program = xelatex
% !TeX program = xelatex
%!TEX encoding = UTF-8 Unicode
%----------------------------------------------------------------------------------------
%   Доорх хэсгийг өөрчлөх шаардлагагүй
%----------------------------------------------------------------------------------------
\documentclass[12pt,A4]{report}

\usepackage{fontspec,xltxtra,xunicode}
\setmainfont[Ligatures=TeX]{Times New Roman}
\setsansfont{Arial}

% \usepackage[utf8x]{inputenc}
% \usepackage[mongolian]{babel}
%\usepackage{natbib}
\usepackage{geometry}
%\usepackage{fancyheadings} fancyheadings is obsolete: replaced by fancyhdr. JL
\usepackage{fancyhdr}
\usepackage{float}
\usepackage{afterpage}
\usepackage{graphicx}
\usepackage{amsmath,amssymb,amsbsy}
\usepackage{dcolumn,array}
\usepackage{tocloft}
\usepackage{dics}
\usepackage{nomencl}
\usepackage{upgreek}
\newcommand{\argmin}{\arg\!\min}
\usepackage{mathtools}
\usepackage[hidelinks]{hyperref}

\usepackage{algorithm}
\usepackage{algpseudocode}

\usepackage{listings}
\DeclarePairedDelimiter\abs{\lvert}{\rvert}%
\makeatletter
\usepackage{caption}
\captionsetup[table]{belowskip=0.5pt}
\captionsetup[table]{name=Хүснэгт}
\usepackage{subfiles}

\usepackage{listings}

\usepackage{color}
\definecolor{lightgray}{rgb}{.9,.9,.9}
\definecolor{darkgray}{rgb}{.4,.4,.4}
\definecolor{purple}{rgb}{0.65, 0.12, 0.82}

\lstdefinelanguage{TypeScript}{
  keywords={abstract, any, as, boolean, break, case, catch, class, console, 
    const, continue, debugger, declare, default, delete, do, else, enum, export, 
    extends, false, finally, for, from, function, get, if, implements, import, in, 
    infer, instanceof, interface, keyof, let, module, namespace, never, new, null, 
    number, object, package, private, protected, public, readonly, require, return, 
    set, static, string, super, switch, symbol, this, throw, true, try, type, typeof, 
    undefined, unique, unknown, var, void, while, with, yield, async, await},
  keywordstyle=\color{blue}\bfseries,
  ndkeywords={class, export, boolean, throw, implements, import, this},
  ndkeywordstyle=\color{darkgray}\bfseries,
  identifierstyle=\color{black},
  sensitive=false,
  comment=[l]{//},
  morecomment=[s]{/*}{*/},
  commentstyle=\color{purple}\ttfamily,
  stringstyle=\color{red}\ttfamily,
  morestring=[b]',
  morestring=[b]"
}

\lstset{
   language=TypeScript,
   backgroundcolor=\color{lightgray},
   extendedchars=true,
   basicstyle=\footnotesize\ttfamily,
   showstringspaces=false,
   showspaces=false,
   numbers=left,
   numberstyle=\footnotesize,
   numbersep=9pt,
   tabsize=2,
   breaklines=true,
   showtabs=false,
   captionpos=b
}
\lstdefinelanguage{yaml}{
  keywords={true,false,null,y,n},
  keywordstyle=\color{darkgray}\bfseries,
  basicstyle=\ttfamily\footnotesize,
  sensitive=false,
  comment=[l]{\#},
  morecomment=[s]{/*}{*/},
  commentstyle=\color{purple}\ttfamily,
  stringstyle=\color{blue}\ttfamily,
  morestring=[b]',
  morestring=[b]",
  breaklines=true,
  breakatwhitespace=true
}
\renewcommand{\lstlistingname}{Код}
\renewcommand{\lstlistlistingname}{\lstlistingname ын жагсаалт}

\usepackage{color}
\definecolor{codegreen}{rgb}{0,0.6,0}
\definecolor{codegray}{rgb}{0.5,0.5,0.5}
\definecolor{codepurple}{rgb}{0.58,0,0.82}
\definecolor{backcolour}{rgb}{0.99,0.99,0.99}
 
\lstdefinestyle{mystyle}{
    basicstyle=\ttfamily\small,
    backgroundcolor=\color{backcolour},   
    commentstyle=\color{codegreen},
    keywordstyle=\color{magenta},
    numberstyle=\tiny\color{codegray},
    stringstyle=\color{codepurple},
    %basicstyle=\footnotesize,
    breakatwhitespace=false,         
    breaklines=true,                 
    captionpos=b,                    
    keepspaces=false,                 
    numbers=left,                    
    numbersep=10pt,                  
    showspaces=false,                
    showstringspaces=true,
    showtabs=false,                  
    tabsize=2
}
 
\lstset{style=mystyle, label=DescriptiveLabel} 

\let\oldabs\abs
\def\abs{\@ifstar{\oldabs}{\oldabs*}}
\makenomenclature
\begin{document}


%----------------------------------------------------------------------------------------
%   Өөрийн мэдээллээ оруулах хэсэг
%----------------------------------------------------------------------------------------

% Дипломийн ажлын сэдэв
\title{Бизнесийн байгууллагын ажилтаны гүйцэтгэлийг үнэлэх систем}
% Дипломын ажлын англи нэр
\titleEng{Employee performance evaluation system for business organization}
% Өөрийн овог нэрийг бүтнээр нь бичнэ
\author{Баянжаргалын Энх-Амгалан}
% Өөрийн овгийн эхний үсэг нэрээ бичнэ
\authorShort{Б. Энх-Амгалан}
% Удирдагчийн зэрэг цол овгийн эхний үсэг нэр
\supervisor{Б. Энхтуул}
% Хамтарсан удирдагчийн зэрэг цол овгийн эхний үсэг нэр
% \cosupervisor{Н. Оюун-Эрдэнэ}

% СиСи дугаар 
\sisiId{21B1NUM0344}
% Их сургуулийн нэр
\university{МОНГОЛ УЛСЫН ИХ СУРГУУЛЬ}
% Бүрэлдэхүүн сургуулийн нэр
\faculty{МЭДЭЭЛЛИЙН ТЕХНОЛОГИ, ЭЛЕКТРОНИКИЙН СУРГУУЛЬ}
% Тэнхимийн нэр
\department{МЭДЭЭЛЭЛ, КОМПЬЮТЕРЫН УХААНЫ ТЭНХИМ}
% Зэргийн нэр
\degreeName{Бакалаврын судалгааны ажил}
% Суралцаж буй хөтөлбөрийн нэр
\programeName{Мэдээллийн технологи(D061303)}
% Хэвлэгдсэн газар
\cityName{Улаанбаатар}
% Хэвлэгдсэн огноо
\gradyear{2025 оны 5 сар}


%----------------------------------------------------------------------------------------
%   Доорх хэсгийг өөрчлөх шаардлагагүй
%----------------------------------------------------------------------------------------
%----------------------Нүүр хуудастай хамаатай зүйлс----------------------------
\pagenumbering{roman}
\makefrontpage
\maketitle

\doublespace

% Decleration
\begin{huge}
	\textbf{Зохиогчийн баталгаа}
\end{huge} \\ \ \\
\doublespace
Миний бие \@author \ "\@title" \ сэдэвтэй судалгааны ажлыг гүйцэтгэсэн болохыг зарлаж дараах зүйлсийг баталж байна:
\begin{itemize}
	\item Ажил нь бүхэлдээ эсвэл ихэнхдээ Монгол Улсын Их Сургуулийн зэрэг горилохоор дэвшүүлсэн болно.
	\item Энэ ажлын аль нэг хэсгийг эсвэл бүхлээр нь ямар нэг их, дээд сургуулийн зэрэг горилохоор оруулж байгаагүй.
	\item Бусдын хийсэн ажлаас хуулбарлаагүй, ашигласан бол ишлэл, зүүлт хийсэн.
	\item Ажлыг би өөрөө (хамтарч) хийсэн ба миний хийсэн ажил, үзүүлсэн дэмжлэгийг дипломын ажилд тодорхой тусгасан.
	\item Ажилд тусалсан бүх эх сурвалжид талархаж байна.
\end{itemize}
\

Гарын үсэг: \underline{\hspace{5cm}}

Огноо: 	\ \ \underline{\hspace{3cm}}

% Гарчгийг автоматаар оруулна
\setcounter{tocdepth}{1}
\tableofcontents

% Зургийн жагсаалтыг автоматаар оруулна
\listoffigures

% Хүснэгтийн жагсаалтыг автоматаар оруулна
\listoftables

% Кодын жагсаалтыг автоматаар оруулна
\lstlistoflistings

% This puts the word "Page" right justified above everything else.
\newpage
%% \addtocontents{lof}{Зураг~\hfill Хуудас \par}
\newpage
%% \addtocontents{lot}{Хүснэгт~\hfill Хуудас \par}

\renewcommand{\cftlabel}{Зураг}


\doublespace
\pagenumbering{arabic}

% Удиртгалыг оруулж ирэх ба abstract.tex файлд удиртгалаа бичнэ
\begin{abstract}
	\setcounter{secnumdepth}{0}
	
	Бизнесийн байгууллагуудын өрсөлдөх чадвар, амжилт нь ажилтнуудын гүйцэтгэлээс ихээхэн хамаардаг. 
    Ажилтны гүйцэтгэлийг үнэлэх нь байгуулла зорилгодоо хүрэх, бүтээмжийг нэмэгдүүлэхэд чухал үүрэгтэй. 
    Энэхүү судалгааны ажлын зорилго нь бизнесийн байгууллагад зориулсан ажилтны гүйцэтгэлийн үнэлгээний системийг вебд суурилан бүтээхэд оршино.

    Энэ хүрээнд Next.js болон Golang хэл дээр суурилсан веб апп-ийг хөгжүүлсэн бөгөөд уг систем нь төслийн удирдлага, 
    даалгаврын менежмент, ажилтны гүйцэтгэлийн үнэлгээний систем зэргийг нэгтгэсэн болно. Тус систем нь удирдлага болон 
    ажилтнуудын хамтын ажиллагааг дэмжиж, даалгаврын хуваарилалт, гүйцэтгэлийн хяналт, үнэлгээний процессыг автоматжуулан, 
    илүү үр дүнтэй, шударга системийг бий болгохыг зорьдог. Судалгаагаар энэхүү системийн онолын загвар, хэрэгжилт, 
    удирдлагын арга барилд үзүүлэх нөлөөг авч үзнэ.
    
    Энэхүү ажлын үр дүнд бизнесийн байгууллагын удирдах албан тушаалтан болон хүний нөөцийн мэргэжилтнүүдэд ажилтны чадавхийг нээн илрүүлэх, 
    гүйцэтгэлийг дээшлүүлэхэд чиглэсэн шийдвэр гаргалтанд дэмжлэг үзүүлэхэд технологийн дэвшилтийг ашиглахад оршино.

	%	\setcounter{secnumdepth}{0} reverse this command
	\setcounter{secnumdepth}{2}

\end{abstract}
\addcontentsline{toc}{part}{БҮЛГҮҮД}

%----------------------------------------------------------------------------------------
%   Дипломын үндсэн хэсэг эндээс эхэлнэ
%----------------------------------------------------------------------------------------
% Шинэ бүлэг
\chapter{СИСТЕМИЙН ТАНИЛЦУУЛГА}
\subfile{src/chapters/chapter1/chapter1}

\chapter{СИСТЕМИЙН СУДАЛГАА}
\subfile{src/chapters/chapter2/chapter2}

\chapter{СИСТЕМИЙН ШИНЖИЛГЭЭ}
\subfile{src/chapters/chapter3/chapter3}

% \chapter{СИСТЕМИЙН ЗОХИОМЖ}
% \subfile{src/chapters/chapter4/chapter4}

% \chapter{ХЭРЭГЖҮҮЛЭЛТ}
% \subfile{src/chapters/chapter5/chapter5}

% \chapter{ДҮГНЭЛТ}





% \chapter{Тайлан боловсруулах зөвлөмж}
% \subfile{writing.tex}


%----------------------------------------------------------------------------------------
%   Дүгнэлт эндээс эхэлнэ
%----------------------------------------------------------------------------------------

%----------------------------------------------------------------------------------------
%   Дипломын номзүй, хавсралтын хэсэг эндээс эхэлнэ
%----------------------------------------------------------------------------------------

\singlespace
\addcontentsline{toc}{part}{НОМ ЗҮЙ}
\begin{thebibliography}{}
	% Ашигласан материалыг эндээс оруулна
	\bibitem{AES}
	Daemen, J., \& Rijmen, V. (2002). "The Design of Rijndael: AES - The Advanced Encryption Standard." Springer. p.1-2.
	\bibitem{intro_crypo}
	Д. Гармаа (2022). "Криптографын үндэс." Улаанбаатар хот.
	\bibitem{modern_crypto}
	Bellare, Mihir; Rogaway, Phillip (11 May 2005), Introduction to Modern Cryptography (Lecture notes), archived (PDF) from the original on 2023-10-30, chapter 3.
	\bibitem{РСА (RSA)}
	Simmons, G. J. (2022, December 29). РСА (RSA) encryption. Encyclopedia Britannica. https://www.britannica.com/topic/РСА (RSA)-encryption

	\bibitem{rsa250factoring}
Boudot, F., Gaudry, P., Guillevic, A., Heninger, N., Thomé, E., \& Zimmermann, P. (2020, February). A 829-bit factorization. Retrieved from \url{https://members.loria.fr/PZimmermann/records/factor.html}

\bibitem{RSAvsECC}
Mahto, Dindayal; YADAV, DILIP. (2017). RSA and ECC: A comparative analysis. International Journal of Applied Engineering Research, Vol. 12, pp. 9053-9061.

\end{thebibliography}


%----------------------------------------------------------------------------------------
%   Хавсралтууд эндээс эхэлнэ
%----------------------------------------------------------------------------------------
\appendix
\addcontentsline{toc}{part}{ХАВСРАЛТ}

% Хавсралтын нэр. Хавсралт гэдэг үг агуулахгүй
% \chapter{Шинжилгээ зохиомж}
% Хавсралтын агуулга

% Хавсралтын нэр. Хавсралт гэдэг үг агуулахгүй
% \chapter{Кодын хэрэгжүүлэлт}

% \lstinputlisting[language=Typescript, caption=tRPC тохиргоо,label=lst:trpc,frame=single]{src/code/trpc.ts}

% \lstinputlisting[language=yaml, caption=Docker Compose ,label=lst:docker-compose,frame=single]{src/code/docker-compose.yml}

\chapter{Back-End хэрэгжүүлэлт}

\par Өгөгдлийн санд байрлах бүхий л өгөгдлүүд рүү хандах API-ууд
\begin{lstlisting}[language=, caption=, frame=single]
    
\end{lstlisting}

\end{document}
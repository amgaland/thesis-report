%!TEX TS-program = xelatex
% !TeX program = xelatex
%!TEX encoding = UTF-8 Unicode
%----------------------------------------------------------------------------------------
%   Доорх хэсгийг өөрчлөх шаардлагагүй
%----------------------------------------------------------------------------------------
\documentclass[12pt,A4]{report}

\usepackage{fontspec,xltxtra,xunicode}
\setmainfont[Ligatures=TeX]{Times New Roman}
\setsansfont{Arial}

% \usepackage[utf8x]{inputenc}
% \usepackage[mongolian]{babel}
%\usepackage{natbib}
\usepackage{geometry}
%\usepackage{fancyheadings} fancyheadings is obsolete: replaced by fancyhdr. JL
\usepackage{fancyhdr}
\usepackage{float}
\usepackage{afterpage}
\usepackage{graphicx}
\usepackage{amsmath,amssymb,amsbsy}
\usepackage{dcolumn,array}
\usepackage{tocloft}
\usepackage{dics}
\usepackage{nomencl}
\usepackage{upgreek}
\newcommand{\argmin}{\arg\!\min}
\usepackage{mathtools}
\usepackage[hidelinks]{hyperref}

\usepackage{algorithm}
\usepackage{algpseudocode}

\usepackage{listings}
\DeclarePairedDelimiter\abs{\lvert}{\rvert}%
\makeatletter
\usepackage{caption}
\captionsetup[table]{belowskip=0.5pt}
\captionsetup[table]{name=Хүснэгт}
\usepackage{subfiles}

\usepackage{listings}

\usepackage{color}
\definecolor{lightgray}{rgb}{.9,.9,.9}
\definecolor{darkgray}{rgb}{.4,.4,.4}
\definecolor{purple}{rgb}{0.65, 0.12, 0.82}

\lstdefinelanguage{TypeScript}{
  keywords={abstract, any, as, boolean, break, case, catch, class, console, 
    const, continue, debugger, declare, default, delete, do, else, enum, export, 
    extends, false, finally, for, from, function, get, if, implements, import, in, 
    infer, instanceof, interface, keyof, let, module, namespace, never, new, null, 
    number, object, package, private, protected, public, readonly, require, return, 
    set, static, string, super, switch, symbol, this, throw, true, try, type, typeof, 
    undefined, unique, unknown, var, void, while, with, yield, async, await},
  keywordstyle=\color{blue}\bfseries,
  ndkeywords={class, export, boolean, throw, implements, import, this},
  ndkeywordstyle=\color{darkgray}\bfseries,
  identifierstyle=\color{black},
  sensitive=false,
  comment=[l]{//},
  morecomment=[s]{/*}{*/},
  commentstyle=\color{purple}\ttfamily,
  stringstyle=\color{red}\ttfamily,
  morestring=[b]',
  morestring=[b]"
}

\lstset{
   language=TypeScript,
   backgroundcolor=\color{lightgray},
   extendedchars=true,
   basicstyle=\footnotesize\ttfamily,
   showstringspaces=false,
   showspaces=false,
   numbers=left,
   numberstyle=\footnotesize,
   numbersep=9pt,
   tabsize=2,
   breaklines=true,
   showtabs=false,
   captionpos=b
}
\lstdefinelanguage{yaml}{
  keywords={true,false,null,y,n},
  keywordstyle=\color{darkgray}\bfseries,
  basicstyle=\ttfamily\footnotesize,
  sensitive=false,
  comment=[l]{\#},
  morecomment=[s]{/*}{*/},
  commentstyle=\color{purple}\ttfamily,
  stringstyle=\color{blue}\ttfamily,
  morestring=[b]',
  morestring=[b]",
  breaklines=true,
  breakatwhitespace=true
}
\renewcommand{\lstlistingname}{Код}
\renewcommand{\lstlistlistingname}{\lstlistingname ын жагсаалт}

\usepackage{color}
\definecolor{codegreen}{rgb}{0,0.6,0}
\definecolor{codegray}{rgb}{0.5,0.5,0.5}
\definecolor{codepurple}{rgb}{0.58,0,0.82}
\definecolor{backcolour}{rgb}{0.99,0.99,0.99}
 
\lstdefinestyle{mystyle}{
    basicstyle=\ttfamily\small,
    backgroundcolor=\color{backcolour},   
    commentstyle=\color{codegreen},
    keywordstyle=\color{magenta},
    numberstyle=\tiny\color{codegray},
    stringstyle=\color{codepurple},
    %basicstyle=\footnotesize,
    breakatwhitespace=false,         
    breaklines=true,                 
    captionpos=b,                    
    keepspaces=false,                 
    numbers=left,                    
    numbersep=10pt,                  
    showspaces=false,                
    showstringspaces=true,
    showtabs=false,                  
    tabsize=2
}
 
\lstset{style=mystyle, label=DescriptiveLabel} 

\let\oldabs\abs
\def\abs{\@ifstar{\oldabs}{\oldabs*}}
\makenomenclature
\begin{document}


%----------------------------------------------------------------------------------------
%   Өөрийн мэдээллээ оруулах хэсэг
%----------------------------------------------------------------------------------------

% Дипломийн ажлын сэдэв
\title{Крифтографын зарим алгоритм, программ}
% Дипломын ажлын англи нэр
\titleEng{Some algorithms and programs for cryptography}
% Өөрийн овог нэрийг бүтнээр нь бичнэ
\author{Даянгийн Балжинням}
% Өөрийн овгийн эхний үсэг нэрээ бичнэ
\authorShort{Д. Балжинням}
% Удирдагчийн зэрэг цол овгийн эхний үсэг нэр
\supervisor{Д. Гармаа}
% Хамтарсан удирдагчийн зэрэг цол овгийн эхний үсэг нэр
% \cosupervisor{Н. Оюун-Эрдэнэ}

% СиСи дугаар 
\sisiId{20B1NUM0563}
% Их сургуулийн нэр
\university{МОНГОЛ УЛСЫН ИХ СУРГУУЛЬ}
% Бүрэлдэхүүн сургуулийн нэр
\faculty{МЭДЭЭЛЛИЙН ТЕХНОЛОГИ, ЭЛЕКТРОНИКИЙН
	СУРГУУЛЬ}
% Тэнхимийн нэр
\department{МЭДЭЭЛЭЛ, КОМПЬЮТЕРЫН УХААНЫ ТЭНХИМ}
% Зэргийн нэр
\degreeName{Бакалаврын судалгааны ажил}
% Суралцаж буй хөтөлбөрийн нэр
\programeName{Программ хангамж(D061302)}
% Хэвлэгдсэн газар
\cityName{Улаанбаатар}
% Хэвлэгдсэн огноо
\gradyear{2023 оны 12 сар}


%----------------------------------------------------------------------------------------
%   Доорх хэсгийг өөрчлөх шаардлагагүй
%----------------------------------------------------------------------------------------
%----------------------Нүүр хуудастай хамаатай зүйлс----------------------------
\pagenumbering{roman}
\makefrontpage
\maketitle

\doublespace

% Decleration
\begin{huge}
	\textbf{Зохиогчийн баталгаа}
\end{huge} \\ \ \\
\doublespace
Миний бие \@author \ "\@title" \ сэдэвтэй судалгааны ажлыг гүйцэтгэсэн болохыг зарлаж дараах зүйлсийг баталж байна:
\begin{itemize}
	\item Ажил нь бүхэлдээ эсвэл ихэнхдээ Монгол Улсын Их Сургуулийн зэрэг горилохоор дэвшүүлсэн болно.
	\item Энэ ажлын аль нэг хэсгийг эсвэл бүхлээр нь ямар нэг их, дээд сургуулийн зэрэг горилохоор оруулж байгаагүй.
	\item Бусдын хийсэн ажлаас хуулбарлаагүй, ашигласан бол ишлэл, зүүлт хийсэн.
	\item Ажлыг би өөрөө (хамтарч) хийсэн ба миний хийсэн ажил, үзүүлсэн дэмжлэгийг дипломын ажилд тодорхой тусгасан.
	\item Ажилд тусалсан бүх эх сурвалжид талархаж байна.
\end{itemize}
\

Гарын үсэг: \underline{\hspace{5cm}}

Огноо: 	\ \ \underline{\hspace{3cm}}

% Гарчгийг автоматаар оруулна
\setcounter{tocdepth}{1}
\tableofcontents

% Зургийн жагсаалтыг автоматаар оруулна
\listoffigures

% Хүснэгтийн жагсаалтыг автоматаар оруулна
\listoftables

% Кодын жагсаалтыг автоматаар оруулна
\lstlistoflistings

% This puts the word "Page" right justified above everything else.
\newpage
%% \addtocontents{lof}{Зураг~\hfill Хуудас \par}
\newpage
%% \addtocontents{lot}{Хүснэгт~\hfill Хуудас \par}

\renewcommand{\cftlabel}{Зураг}


\doublespace
\pagenumbering{arabic}

% Удиртгалыг оруулж ирэх ба abstract.tex файлд удиртгалаа бичнэ
\begin{abstract}
	\setcounter{secnumdepth}{0}
	Бизнесийн байгууллагуудын хувьд ажилчдын гүйцэтгэлийг үнэлэх нь байгууллагын амжилт, хөгжлийн чухал хэсэг юм. Уламжлалт арга буюу гар ажиллагаатай үнэлгээний систем нь цаг хугацаа их шаардлага гаргаж, алдаа гарах эрсдэлтэй байдаг. Иймээс веб-д суурилсан систем ашиглан автоматжуулсан, ил тод, үр дүнтэй гүйцэтгэлийн үнэлгээний систем хөгжүүлэх шаардлага бий болсон. Энэхүү тайланд Golang, Next.js, Docker зэрэг технологийг ашиглан хэрхэн ийм системийг хөгжүүлэх талаар өгүүлнэ. Тайлангийн зорилго нь системийн онолын үндэс, шаардлага, зохиомж, хэрэгжүүлэлтийг тодорхойлж, практикт хэрэглэх боломжтой шийдэл санал болгох явдал юм.

	%	\setcounter{secnumdepth}{0} reverse this command
	\setcounter{secnumdepth}{2}

\end{abstract}

\addcontentsline{toc}{part}{БҮЛГҮҮД}

%----------------------------------------------------------------------------------------
%   Дипломын үндсэн хэсэг эндээс эхэлнэ
%----------------------------------------------------------------------------------------
% Шинэ бүлэг
\subfile{src/chapters/chapter1}
% Энэ бүлэгт бакалаврын судалгааны ажлын хүрээнд тавьсан зорилго, зорилтын хүрээнд ижил 
% төстэй системийн судалгаа мөн уг системийг хөгжүүлэхэд ашиглагдах технологиудын судалгаа
% хийв.
\subfile{onol.tex}

\subfile{samesystem.tex}

\subfile{techsudalgaa.tex}

\section{Бүлгийн дүгнэлт}
\subfile{dugnelt.tex}


\subfile{user.tex}

\subfile{functional.tex}

% \section{Технологийн шаардлага}
% \subfile{tech.tex}

% \section{Функциональ загвар}
% \subfile{funcdesign.tex}

% \section{Функциональ загварын тодорхойлолт}
% \subfile{funcdesigndet.tex}


% \section{Бүлгийн дүгнэлт}
% \subfile{dugnelt.tex}





3.1 Системийн хэрэглэгчид 
Системийн шаардлага
    UI/UX шаардлага 
    3.2 Функциональ шаардлага
    3.3 Технологийн шаардлага
Системийн загвар
    4.1 Системийн архитектур
    4.2 Үйл ажиллагааны загвар
    4.4 Системийн статик загвар
    4.5 Өгөгдлийн сангийн загвар
    4.6 Дэлгэцийн зохиомж



% \chapter{Тайлан боловсруулах зөвлөмж}
% \subfile{writing.tex}


%----------------------------------------------------------------------------------------
%   Дүгнэлт эндээс эхэлнэ
%----------------------------------------------------------------------------------------
\conclusion{Дүгнэлт}
Энэхүү судалгааны ажлаар дэлхий нийтэд ашиглагдаж буй криптографын зарим алгоритмуудыг судалж хэрэгжүүлсэн билээ. Энэхүү судалж суралцсан мэдлэгээ ашиглан практикт олон улсын стандартад нийцсэн үүлэн технологит суурилсан тоон гарын үсгийн системийн бүтээхийг зорилоо. Үр дүнд нь хамгийн орчин үеийн шинэлэг үүлэн технологиудтай танилцсан ба, бүтээгдэхүүний шаардлагыг гаргаж эх кодыг үүсгэхээс эхлээд эцсийн хэрэглэгчид хүрэх, чанарын шаардлагыг хангаж ачаалал даахуйц системийг бүтээлээ.

Энэхүү систем нь үүлэн технологит суурилсан гэдгээрээ Монгол улсад анхдагч болж байгаа юм. Энэ төрлийн систем нь нууцлалыг маш өндөр түвшинд хангаж байх нь хамгийн чухал байсан ба нийт хөгжүүлэлтийн ажлын дийлэнх цаг нь үүлэн технологийг судлахад зарцуулагдсан билээ. Үүнээс гадна, хэрэглэгчийн утсанд цагаас хамааран өөрчлөгдөх нууц үг тохируулж, гарын үсгийг нь тухайн хэрэглэгчийн нууц үгээр дахин шифрлэж хоёроос гурван шатны хамгаалалтыг нэмсэн юм. Олон улсын стандартад нийцсэн хэлбэрээр бичиг баримтын баталгаажуулах нь нэн төвөгтэй байсан ба нээлттэй эхийн PDF дээр л зөвхөн бүрэн утгаараа ажиллаж байгаа юм. Энэ нь PDF файл цаанаа гарын үсэг зурах хэсэгтэй байдагтай холбоотой. 
Мөн түүнчлэн систем дээрээ сонгодог крифтографын алгоритмуудыг уншсан судалгаанаасаа ямар нэгэн сан ашиглахгүйгээр хэрэгжүүлэхийг оролдсон бөгөөд, энэ нь нэн төвөгтэй ажил байсан. Иймд дугуйг дахин зохион бүтээх шаардлагагүй гэдэгчлэн олон жилийн туршид шалгагдаж, стандарт хангасан сан, кодыг ашиглан хөгжүүлэх нь зүйтэй.
Цаашлаад блокчейн технологийг ашиглан бүр ч илүү найдвартай, нийтэд нээлтэй систем болох боломжтой гэж харж байна.
%----------------------------------------------------------------------------------------
%   Дипломын номзүй, хавсралтын хэсэг эндээс эхэлнэ
%----------------------------------------------------------------------------------------

\singlespace
\addcontentsline{toc}{part}{НОМ ЗҮЙ}
\begin{thebibliography}{}
	% Ашигласан материалыг эндээс оруулна
	\bibitem{AES}
	Daemen, J., \& Rijmen, V. (2002). "The Design of Rijndael: AES - The Advanced Encryption Standard." Springer. p.1-2.
	\bibitem{intro_crypo}
	Д. Гармаа (2022). "Криптографын үндэс." Улаанбаатар хот.
	\bibitem{modern_crypto}
	Bellare, Mihir; Rogaway, Phillip (11 May 2005), Introduction to Modern Cryptography (Lecture notes), archived (PDF) from the original on 2023-10-30, chapter 3.
	\bibitem{РСА (RSA)}
	Simmons, G. J. (2022, December 29). РСА (RSA) encryption. Encyclopedia Britannica. https://www.britannica.com/topic/РСА (RSA)-encryption

	\bibitem{rsa250factoring}
Boudot, F., Gaudry, P., Guillevic, A., Heninger, N., Thomé, E., \& Zimmermann, P. (2020, February). A 829-bit factorization. Retrieved from \url{https://members.loria.fr/PZimmermann/records/factor.html}

\bibitem{RSAvsECC}
Mahto, Dindayal; YADAV, DILIP. (2017). RSA and ECC: A comparative analysis. International Journal of Applied Engineering Research, Vol. 12, pp. 9053-9061.

\end{thebibliography}


%----------------------------------------------------------------------------------------
%   Хавсралтууд эндээс эхэлнэ
%----------------------------------------------------------------------------------------
\appendix
\addcontentsline{toc}{part}{ХАВСРАЛТ}

% Хавсралтын нэр. Хавсралт гэдэг үг агуулахгүй
% \chapter{Шинжилгээ зохиомж}
% Хавсралтын агуулга

% Хавсралтын нэр. Хавсралт гэдэг үг агуулахгүй
% \chapter{Кодын хэрэгжүүлэлт}

% \lstinputlisting[language=Typescript, caption=tRPC тохиргоо,label=lst:trpc,frame=single]{src/code/trpc.ts}

% \lstinputlisting[language=yaml, caption=Docker Compose ,label=lst:docker-compose,frame=single]{src/code/docker-compose.yml}


\end{document}